% Finale versie compileren:
% Geef de disable optie mee aan de todonotes package
% Geef de hidelinks optie mee aan hyperref
% Schakel de nag package uit (?)
% Voer makeglossary uit om glossaries bij te werken
% Compileer tweemaal zodat alle referenties up-to-date zijn

% Checks for obsolete LaTeX packages and outdated commands.
\RequirePackage[l2tabu, orthodox]{nag}

\documentclass[a4paper,dutch]{report}

% Font encoding. Support for < and > characters, for example.
% Note that the default font doesn't work with T1. You'll need lmodern or cm-super as well.
\usepackage[T1]{fontenc}
% UTF-8 file input encoding. Support for tons of accents like äëöü and more.
\usepackage[utf8]{inputenc}
% Latin Modern font. Required for T1 font encoding.
\usepackage{lmodern}
% Alternatieve monospaced font. Gebruikt in listings.
\usepackage[scaled]{beramono}
\usepackage[dutch]{babel}
\usepackage{a4wide} % Obsolete & buggy (supposedly). See alternatives below.
% \usepackage{fullpage}
% \usepackage[a4paper,margin=1cm,footskip=.5cm]{geometry}
% Fancy headers
\usepackage{fancyhdr}
% Alternative to fancyhdr
% \usepackage{titleps}
% Support for importing images.
\usepackage{graphicx}
% Fix links to figures from hyperref
\usepackage[hypcap]{caption}
% Afkortingen worden apart afgedrukt
% Gebruik nomain om enkel afkortingen te gebruiken
\usepackage[acronym]{glossaries}
% Voegt bibliografie/referenties, inhoudstafel en figurenlijst toe aan inhoudstafel.
%\usepackage{tocbibind}
\usepackage[backend=biber]{biblatex}
% Code listings/samples
\usepackage{listings}
% Extra colors for the todonotes package (or in general)
\usepackage[dvipsnames]{xcolor}
% Todo notes.
%TODO: Add the disable option for todo notes for final release.
\usepackage[color=YellowGreen]{todonotes}
% Replace new paragraph indentation with whitespace.
\usepackage[parfill]{parskip}
% Easy inserting of external PDF files.
\usepackage{pdfpages}
% More configuration options for working with appendices.
\usepackage[toc]{appendix}
% Additional commands to enhance the quality of tables in LaTeX.
\usepackage{booktabs}

% All kinds of references become clickable hyperlinks in PDF. Should be loaded late.
\usepackage{hyperref}
% Clever references. Automatically adds "figure, table, ..." based on context to references. Load as last package to change referencing system.
% nameinlink includes "figure, table, ..." in the hyperref link as opposed to only the number. For hyperref you'd need the \autoref command.
% noabbrev never uses abbreviations. Instead of "fig." you'll get "figure".
\usepackage[nameinlink,noabbrev]{cleveref}
% Micro-typographic extensions: character protrusion and font expansion. Load after loading fonts.
% http://www.ctan.org/tex-archive/macros/latex/contrib/microtype
% http://www.khirevich.com/latex/microtype/
\usepackage{microtype}


% Configuration

% Don't uppercase titles in page headers.
\lhead{\nouppercase{\rightmark}}
\rhead{\nouppercase{\leftmark}}

% Vervang de meervoud suffix voor afkortingen van de glossaries package door 's
\renewcommand{\acrpluralsuffix}{'s}

% Change listings settings.
\lstset{
	tabsize=4,				% Change tabsize from 8 spaces to 4 for listings.
	captionpos=b,			% Change the caption position from top to bottom.
	frame=single,			% Adds a frame around code listings.
	breaklines=true,		% Activates automatic line breaking of long lines.
	breakatwhitespace=true,	% Only breaks at whitespaces.
	basicstyle=\footnotesize\ttfamily	% Use a smaller font size.
}
% Change the caption label for listings.
\renewcommand{\lstlistingname}{Codefragment}
% Change the header name for the list of listings.
\renewcommand{\lstlistlistingname}{Codefragmenten}
% To make sure code doesn't split over a page break use the float option, optionally with the =h parameter.
% Or alternatively with a minipage, see: https://tex.stackexchange.com/questions/18492/how-can-i-ensure-that-a-listing-is-not-going-to-be-split

% Figuurnaam en listingnaam voor gebruik in zinnen.
%\newcommand{\lstlistingnamesentence}{\MakeLowercase{\lstlistingname}}
\newcommand{\lstlistingnamesentence}{{\huge\textbf{VERWIJDER MIJ}}}
%\newcommand{\figurenamesentence}{\MakeLowercase{\figurename}}
\newcommand{\figurenamesentence}{{\huge\textbf{VERWIJDER MIJ}}}
%\newcommand{\tablenamesentence}{\MakeLowercase{\tablename}}
\newcommand{\tablenamesentence}{{\huge\textbf{VERWIJDER MIJ}}}

% Hernoem de naam voor bijlagen in de inhoudstafel. Standaard: Appendices.
\renewcommand{\appendixtocname}{Bijlagen}

% Er is nog geen vertaling voor listings, dus kun je ze hiermee overschrijven.
\crefname{listing}{codefragment}{codefragmenten}
\Crefname{listing}{Codefragment}{Codefragmenten}

% Mogelijke optie om rode kaders rond references te verwijderen.
% Kleurt references donkerrood -> enkel voor online gebruik, niet om af te drukken.
%\hypersetup{
%	colorlinks,
%	linkcolor={red!50!black},
%	citecolor={blue!50!black},
%	urlcolor={blue!80!black}
%}


% Load .bib files / referenties inladen
\addbibresource{referenties.bib}

% Geen paginanummering en hoofding.
\pagestyle{empty}

% Generate the glossary, must be used before glossary entries
\makeglossaries

% Include wordt afgeraden voor glossaries! Bovendien werken includes toch niet in de preamble.
% Woordenlijst

\newglossaryentry{arptabel}
{
	name=ARP-tabel,
	description={ARP staat voor het Address Resolution Protocol.
				Met behulp van ARP kan een toestel het unieke hardware-adres (MAC-adres) te weten komen dat bij een IP-adres hoort van een ander toestel.\cite{arp-nlwiki}}
}


% Afkortingen

\newacronym{mdb}{MDB}{Manageable Objects Database}
\newacronym{oid}{OID}{Object Identifier}
\newacronym{mib}{MIB}{Management Information Base}
\newacronym{pdu}{PDU}{Protocol Data Unit}

% Afkortingen met afwijkend meervoud

%\newacronym{nms}{NMS}{Network Management Systeem}
\newglossaryentry{nms}
{
	name=NMS,
	description={Network Management Systeem},
	first={Network Management Systeem (NMS)},
	plural=NMS'en,
	longplural={Network Management Systemen},
	type=\acronymtype
}

% Extra language definitions for the listings package to list code fragments
%% ASN.1 / MIB files
\lstdefinelanguage[]{asn.1}
{keywords=
	{OBJECT-TYPE,SYNTAX,SEQUENCE,OF,ACCESS,STATUS,DESCRIPTION,INDEX,INTEGER,DisplayString,Gauge,PhysAddress,TimeTicks,Counter,OBJECT,IDENTIFIER},
	alsoletter=-,
	sensitive=true % Case sensitive?
}[keywords]


% % % % % %
% Eerste keer dat een ~vreemd~ woord wordt gebruikt -> cursief, daarna normaal.
% % % % % %


\begin{document}

% List of todo's (added as the first page)
\listoftodos

% Bedrukte kaft met titel (licht karton)
\includepdf{titelblad.pdf}

% Blanco blad
\newpage
\null
\thispagestyle{empty}
\newpage

% Titelblad. Moest de kaft vervagen, bevat deze alle belangrijke info net als de kaft.
\includepdf{titelblad.pdf}

% Woord vooraf
% Dankbetuigingen
% Sahel, my saving knight met zijn LXC opstelling op de Virtual Wall
% Bart voor zijn tripjes naar Oostende met de was en lege DVD's :D
\newpage
\noindent \textbf{\huge Woord vooraf}

% Abstract Nederlands (facultatief) - Abstract Engels (facultatief)
% ~paper: kernwoorden, samenvatting
%\newpage
%\noindent \textbf{\huge Abstract}

\chapter*{Abstract}
\addcontentsline{toc}{chapter}{Abstract}

Samenvatting (Engels \& Nederlands)

Kernwoorden

\todo[inline]{Zie inleiding benchmarks \& experimenten}

% Inhoudsopgave
\tableofcontents

% Vanaf hier wel paginanummering en hoofding.
\pagestyle{fancy}

% Lijst met tabellen (facultatief)
\listoftables

% Lijst met figuren (facultatief)
\listoffigures

% Lijst van codefragmenten
\lstlistoflistings

% Woordenlijst
\printglossary[title=Woordenlijst]

% Lijst met afkortingen (facultatief)
% ENKEL de afkortingen die verondersteld worden als niet gekend door de lezer. Dus bv. niet DNA, ISO, ...
\printacronyms[title=Lijst van gebruikte afkortingen]

% Inleiding
% (duidelijke vraagstelling, aankondiging van structuur …)
% (duidelijke vraagstelling, aankondiging van structuur …)
% Engelse termen worden niet in cursief geplaatst (zie boek)
% Geen volledig hoofdstuk, max 3 pg's.
% Zie pg 51 boek

% Meer achtergrond voor niet-deskundige lezer
% Stand van onderzoek voor het eindwerk
% Probleemstelling
% Begrenzing van het thema
% Onderzoeksvraag
% Verantwoording van de keuze !!!
% Doelstelling
% Voornaamste bronnen van informatie
% Redenen van het onderzoek
% Methode van het onderzoek
% Hoofdlijnen van het eindwerk en hun algemene samenhang, zonder de inhoudsopgave te herhalen

\chapter{Inleiding}
\todo[inline]{Bespreek de inhoud. Duidelijke vraagstelling, aankondiging van structuur (zie source scriptie).}


\section{Situering}
Bestaande (open-source) softwaretools voor het opvragen van netwerkinformatie via SNMP beschikken over te weinig intelligentie om
grootschalige netwerken te ondervragen.
Hierdoor is het moeilijk om op frequente basis datamining van het netwerk te doen om bijvoorbeeld de netwerkconnectiviteit of netwerkroutering in kaart te brengen.
Het efficiënt kunnen nagaan van configuratiefouten van grootschalige Ethernet- en IP-netwerken is nochtans een must voor de netwerkbeheerder.

Bij het bedrijf NetworkMining, dat zich profileert als een onafhankelijke softwareleverancier voor transportnetwerken, wordt er aan netwerkbevraging
gedaan via het SNMP-protocol met behulp van een intern ontwikkelde tool. Bij het aanvatten van de masterproef beschikte deze tool ook nog niet over veel intelligentie.
Gegeven een lijst van toestellen en op te vragen gegevens ging ze iteratief elk gegeven gaan opvragen aan de toestellen zonder
er verder bij stil te staan wat er juist opgevraagd werd.
Ook voor het bijhouden van historische data kan extra intelligentie goed van pas komen.


\section{Probleemstelling} %TODO: fix gls{nms}
Fabrikanten van routers en switches voorzien nu al network management systemen die het leven van een netwerkbeheerder makkelijker maken.
Deze systemen kunnen onder andere geaggregeerde informatie van de verschillende netwerkelementen rapporteren.
Het probleem hierbij echter is dat deze managementsystemen enkel werken voor netwerkelementen van dezelfde fabrikant.
Grootschalige netwerken bestaan echter uit apparatuur van verschillende fabrikanten,
typisch bepaald door het afwegen van de kostprijs en features die een bepaalde fabrikant aanbiedt op het moment dat een netwerk uitgebreid wordt. %TODO: lange zin
Elke fabrikant biedt zo wel een eigen network management systeem aan voor de eigen apparaten.
Maar ook het raadplegen van deze systemen is niet uniform: er wordt gebruik gemaakt van verschillende API's en technologieën zoals XML SOAP en CORBA.
Ze bieden bovendien niet altijd alle informatie aan die de netwerkbeheerder wenst...
Vandaar dat ervoor geopteerd werd om gebruik te maken van het SNMP protocol voor het opvragen van netwerkinformatie.
Dit protocol wordt wel door apparatuur van alle fabrikanten ondersteund en biedt een enigszins uniform alternatief.
Het SNMP-protocol biedt wel niet dezelfde aggregatiemogelijkheden als een network management systeem.
In de plaats daarvan gaat het over ruwe informatie van individuele netwerkcomponenten die verder verwerkt en geaggregeerd moet worden.


\section{Doelstelling}
De SNMP Data Retriever die hierboven werd beschreven, wordt vandaag enkel gebruikt op kleine- tot middelgrote netwerken.
De bedoeling is om deze software ook te kunnen inzetten op grootschalige netwerken waar ze haar nut het meest kan bewijzen.
Men kan echter niet zondermeer de bestaande software inzetten op die grote netwerken, er komen tal van zaken een rol spelen
waar dat bij kleinere netwerken niet het geval was. Deze moet men dan vooral zoeken in de richting van performantieproblemen.
Zoals gezegd is de software niet voorzien van veel intelligentie of zelfs ontwikkeld met performantie in het achterhoofd.
De bedoeling van de masterproef bestaat erin om de schaalbaarheid te onderzoeken van de SNMP-bevragingen met de bestaande software.
Hierbij moet er gezocht worden naar mogelijke bottlenecks die zich voordoen.
Dit kan gaan om de CPU van de client, bandbreedteproblemen, de databank die de opgevraagde gegevens moet opslaan die niet kan volgen of
het netwerkelement zelf die niet snel genoeg is.
Eens de bottlenecks geïdentificeerd zijn moet er gezocht worden naar oplossingen om de bottlenecks te verhelpen.
Denk aan aanpassingen aan de software zoals het implementeren van multithreading, gelijktijdig gebruik van meerdere SNMP clients of
het opzetten van een databankcluster.
Om te zien hoe effectief de oplossingen zijn zal er ook een testmethode/benchmark opgesteld moeten worden om dit na te meten. 


\section{Mogelijke uitbreidingen}
\todo[inline]{
Misschien lager in de scriptie plaatsen vermits de uitbreidingen niet geïmplementeerd zijn.
Misschien meer naar het einde toe...}
Als uitbreiding kan de bestaande software voorzien worden van enige intelligentie bij het opvragen van netwerkinformatie.
Het is zo dat sommige gegevens zeer dynamisch zijn maar andere gegevens kunnen quasi statisch zijn.
Denk bijvoorbeeld aan temperatuurmetingen en een overzicht van alle netwerkinterfaces die aanwezig zijn in een toestel.
Het eerste gegeven verandert constant, het laatste haast nooit.
Het is dan ook logisch dat het laatste niet zo vaak opgevraagd moet worden als het eerste.
Zo zouden we kunnen een algoritme of heuristiek ontwikkelen die de veranderlijkheid van gegevens bepaalt en afhankelijk daarvan
beslist hoe vaak dat gegeven opgehaald moet worden.
Gegevens kunnen ook meer of minder belangrijk zijn als andere.
De netwerkbeheerder zou dan ook zelf kunnen de periodiciteit opgeven voor het opvragen van bepaalde gegevens,
afhankelijk van het belang dat de netwerkbeheerder eraan hecht.
Ook voor het bijhouden van historische data kan dit een interessante optie zijn.

Een ander idee is om te zien of de netwerkcomponenten SNMP-requests beantwoorden die via broadcast of
multicast (na het inschrijven op een multicastgroep) verstuurd zijn geweest.
Bij Linux machines is dit bijvoorbeeld wel het geval.
Hierdoor zou het netwerkverkeer om SNMP informatie te verzamelen quasi gehalveerd kunnen worden.
De SNMP retriever moet niet meer elk netwerkelement individueel ondervragen maar ondervraagt ze allemaal tegelijkertijd.
Het netwerkvolume dat gegenereerd wordt door de netwerkcomponenten als antwoord blijft wel even groot,
en afhankelijk van de omvang van het netwerk zou dit ook wel eens een zeer zware belasting voor de retriever kunnen zijn.
De SNMP-retriever moet dan tenslotte de antwoorden van alle netwerkelementen tegelijkertijd kunnen verwerken. %TODO: wordflow met vorige zin
Toch is het zeker een optie die de moeite waard is om te onderzoeken.
Er moet dan ook gekeken worden wat voor aanpassingen er nodig zijn aan het netwerk om dit te ondersteunen:
het inschrijven van de nodes op een multicastgroep, het toestaan van het routeren van multicastverkeer op de routers en het voorzien van bijhorende multicastroutes.

% Engelse termen worden niet in cursief geplaatst (zie boek)


% Corpus, middenstuk, eigenlijke tekst

% SNMP
\chapter{SNMP}

\section{Inleiding}

SNMP staat voor het Simple Network Management Protocol. De naam vertelt je meteen al waarvoor het protocol dient: het beheren van je netwerk.
De 'S' van SNMP betekent niet simpel als in simpel in gebruik (alhoewel het niet moeilijk in gebruik is), maar dat het protocol zo simpel mogelijk gehouden is.
Met SNMP kun je allerhande informatie opvragen van allerhande soorten toestellen. In principe is er geen limiet op hetgeen je kunt opvragen,
maar de functionaliteit om die informatie op te vragen moet wel geïmplementeerd worden. Normaal gezien gebeurt dat door de fabrikant van het toestel.

Er zijn maar een paar voorwaarden om met behulp van SNMP informatie van een toestel op te kunnen vragen. Eerst en vooral moet het toestel natuurlijk SNMP ondersteunen.
Ze moet ook de functionaliteit bezitten om de gevraagde informatie te kunnen aanbieden en tenslotte moet de vrager ook kennis hebben van welke informatie
er allemaal opgevraagd kan worden. Zo zal een router niet dezelfde informatie kunnen aanbieden als bijvoorbeeld een switch. Er is wel een minimale
verzameling van gegevens die door alle toestellen ondersteund moet worden om als SNMP-compatibel bestempeld te kunnen worden. Denk hierbij aan de naam van het
systeem, hoe lang het systeem al online is, welke netwerkinterfaces ze heeft, enzovoort.

Er zijn een groot aantal use cases te bedenken voor het gebruik van SNMP maar hier zijn enkele van de belangrijkste.

\begin{itemize}
	\item Inventarisatie: met behulp van discovery procedures en SNMP kun je een overzicht krijgen van alle toestellen aangesloten op het netwerk,
	en vooral hoe ze met elkaar verbonden zijn. Netwerken zijn niet statisch: er worden toestellen toegevoegd en verwijderd. Met SNMP kun je een
	actueel beeld van de aangesloten hardware en netwerktopologie verkrijgen. Dit is zeer handig voor kleine en middelgrote netwerken, maar onmisbaar
	voor grote netwerken!
	
	\item Configuratiebeheer: voor wat betreft de hardwarekant overlapt dit eigenlijk met de inventarisatie. Maar voor de softwarekant kun je met SNMP de
	configuratieinstellingen van toestellen opvragen en controleren. Bijvoorbeeld: voor Windowstoestellen kun je de lijst opvragen van geïnstalleerde updates,
	van routers kun je de routetabel controleren en van switches de \gls{arptabel}.
	
	\item Performantiebeheer: ook de netwerkverzadiging kun je controleren met SNMP. Zo kun je de huidige toestand in de gaten houden,
	onregelmatigheden vaststellen en trends volgen. Op basis van deze trends kunnen dan plannen opgesteld worden voor toekomstige uitbreidingen
	van de netwerkcapaciteit op knelpunten.
\end{itemize}

Voor kleine netwerken kan SNMP het leven van een systeembeheerder een stuk gemakkelijker maken. Om grote netwerken te beheren is SNMP quasi een vereiste.
De conclusie is duidelijk: SNMP zou onderdeel moeten uitmaken van de gereedschapskit van elke systeembeheerder.


% SNMP meer in detail / bouwstenen/onderdelen van SNMP
% Aspecten van SNMP?

\section{SNMP meer in detail}
\todo{Betere titel?}

\subsection{De S van SNMP}
\todo{Betere titel?}

Zoals gezegd staat de 'S' van SNMP voor simple. Hiermee wordt vooral bedoeld dat men SNMP zo simpel mogelijk heeft gehouden.
De reden waarom dit zo is, is tweeledig: eenvoud van implementatie en met oog op performantie. % Voor het toestel, niet het netwerk!!!
Omdat SNMP zo simpel in elkaar steekt, kunnen fabrikanten zonder al te veel moeite SNMP implementeren op hun hardware.
Door de simpliciteit van SNMP moet er bovendien slechts weinig werk verricht worden om SNMP-requests te beantwoorden.
Zodoende kunnen zelfs toestellen met zeer zwakke hardware, zoals embedded systemen, ook SNMP ondersteunen.


% Componenten?
\subsection{Onderdelen van SNMP}
\todo{Betere titel?}

SNMP kent twee soorten toestellen: SNMP-agents en SNMP-managers. In het client-server principe wordt de serverrol verricht door de SNMP-agent
en de client door de SNMP-manager.

\begin{itemize}
	\item De SNMP-agent is een stuk software dat draait op de netwerkcomponenten en informatie bijhoudt over het toestel.
	Als iemand daar om vraagt zal de agent hem ook die informatie verstrekken.
	
	% De GLS Reset zorgt ervoor dat de afkorting zeker languit wordt uitgeschreven
	\item De SNMP-manager, vaak een \glsreset{nms} \gls{nms} genoemd, is de software die de SNMP-requests genereert, verstuurt naar de
	SNMP-agents en uiteindelijk de antwoorden verwerkt. Een \gls{nms} zal meestal niet één maar meerdere SNMP-agents van verschillende toestellen ondervragen.
	Aan de hand van de verkregen informatie kan de \gls{nms} beslissen om verdere acties uitvoeren.
\end{itemize}


\subsection{Uniformiteit}
\todo{Betere titel?}

Een van de grootste voordelen van SNMP is het feit dat ze je een uniforme manier aanbiedt om je netwerk te beheren. Fabrikanten van netwerkapparatuur
bieden graag hun eigen oplossing aan om hun toestellen te beheren, maar het probleem is dat netwerken slechts hoogst uitzonderlijk uit apparatuur
bestaan van slechts een fabrikant. De managementoplossingen van de ene fabrikant werken natuurlijk niet om ook de toestellen van de andere
fabrikant te beheren. Met SNMP zit je dus niet vast aan een fabrikant en kun je op uniforme wijze gans je netwerk beheren.

SNMP biedt maar weinig functionaliteit aan en dat speelt zowel in zijn voordeel als in zijn nadeel. \todo{Te lange zin?} Het voordeel is zoals eerder vermeld het
gemak waarmee fabrikanten SNMP kunnen implementeren, maar het nadeel is het gebrek aan features: met SNMP kun je enkel ruwe data opvragen.
Als je meer uit je data wenst te halen moet je die zelf achteraf verder verwerken. Denk aan het bijhouden van historische data, het combineren
van verschillende gegevens om verbanden te zien, grafieken opmaken of zelfs ganse rapporten opstellen. Die dingen worden wel ondersteund door de
managementoplossingen van fabrikanten, maar bij SNMP is het aan de gebruiker om die data uit de SNMP-gegevens te extraheren.


\subsection{Onderliggende protocols}

SNMP steunt op IP en UDP als transportprotocol. Dankzij de IP laag kan SNMP ook werken op heterogene netwerken.
UDP werd verkozen boven TCP vanwege de eenvoudige werking, wat weer handig is voor low-level netwerkcomponenten.
Bovendien heeft UDP een kleinere impact op het netwerkverkeer dan TCP\cite{moreau}.
Het nadeel van UDP is wel dat er geen bevestiging gebeurt van verstuurde pakketten.
Als er een pakket verloren gaat is het dan aan de \gls{nms} om dit te detecteren en op te vangen.
Normaal zal de \gls{nms} simpelweg de SNMP-query opnieuw versturen.
Voor het versturen en ontvangen van SNMP-berichten wordt er standaard gebruik gemaakt van poort 161.


\section{Object Identifiers}
\todo[inline]{Alinea's opsplitsen.}
Elk mogelijk soort gegeven dat je kunt opvragen van een toestel dat SNMP ondersteunt wordt wereldwijd uniek geïdentificeerd\cite{moreau} door een \gls{oid}.
\glspl{oid} worden hiërarchisch ingedeeld in een boomstructuur, net als bij DNS. De eindpunten van de boom stellen objecten voor en de knooppunten van de boom
worden gebruikt om objecten logisch te groeperen. De namen van de knooppunten bepalen de uiteindelijke \gls{oid}.
\glspl{oid} hebben een dotted-decimal notatie waarbij de knooppunten en het object van elkaar gescheiden worden door punten. %TODO Citatie?
Van links naar rechts wordt de \gls{oid} opgebouwd door het hoogste knooppunt tot het uiteindelijk object.
Een voorbeeld van een \gls{oid} die de uptime van een syteem teruggeeft is iso.org.dod.internet.mgmt.mib-2.system.sysUpTime.
Het stuk van de boomstructuur waarin die \gls{oid} valt kun je ook zien in figuur \ref{boomstructuur}.
De tekstuele notatie van een \gls{oid} valt zoals je ziet nogal lang uit en is moeilijk om te onthouden.
Daarom is het ook mogelijk om van een numerieke notatie gebruik te maken.
Elk knooppunt heeft behalve de tekstuele naam ook een volgnummer waar gebruik van kan gemaakt worden en een veel kortere \gls{oid} oplevert.
De \gls{oid} die hiervoor als voorbeeld werd gegeven wordt dan 1.3.6.1.2.1.1.3.
In de figuur van de boomvoorstelling staat het volgnummer van elk knooppunt ook tussen ronde haken. Eventueel kan er ook gebruik gemaakt worden van een
hybride notatie waarbij afwisselend gebruik kan gemaakt worden van de tekstuele of numerieke identificatie van een knooppunt.
Een mogelijke hybride voorstelling van het vorig voorbeeld is 1.3.6.1.mgmt.1.1.sysUpTime.

\begin{figure}[h]
	\centering
	\includegraphics{figures/snmp/OID_tree}
	\caption{Boomstructuur van SNMP-objecten}
	\label{boomstructuur}
\end{figure}


\section{Management Information Base}
SNMP-agents houden een overzicht bij van alle gegevens die ze moeten bijhouden in een zogeheten \gls{mdb}.
In die \gls{mdb} zit een verzameling van \gls{mib} bestanden.\footnote{De \gls{mib}-bestanden worden eerst gecompileerd naar een binair formaat
alvorens ze in de \gls{mdb} opgeslagen worden.\cite{moreau}}
In die \gls{mib}-bestanden worden de eigenlijke SNMP-objecten gedefiniëerd. Vaak gaat men samenhorende objecten in een \gls{mib}-bestand samen stoppen.
Bijvoorbeeld alle objecten die bij een bepaald protocol horen, of alle objecten die worden geïmplementeerd door een bepaalde fabrikant.
Voor elk object wordt de naam en het volgnummer vastgelegd die zullen gebruikt worden in het OID voor dat object, alsook een beschrijving ervan en
van wat voor soort datastructuur er gebruik gemaakt wordt om de data voor te stellen (bv. een string of integer).

Ook voor \glspl{nms} zijn \gls{mib}-bestanden zeer belangrijk. Want ook zij moeten weten welke gegevens ze precies kunnen opvragen van een bepaalde
agent, en vooral: hoe ze die gegevens moeten interpreteren! Een \gls{nms} zal dus minstens over dezelfde \glspl{mib} moeten beschikken als de
agents die hij ondervraagt. Een groot aantal \glspl{mib} zijn gestandaardiseerd en worden ook standaard meegeleverd met SNMP-software.
Een aantal van die gestandaardiseerde \glspl{mib} moeten ook verplicht ondersteund worden door SNMP-agents om de stempel 'SNMP compatibel' te mogen dragen.
Het MIB-2 knooppunt die je kunt zien in \cref{boomstructuur} is daar een voorbeeld van.
Die minimale set van gegevens zal je dus van ieder SNMP toestel kunnen opvragen.


\section{Tabellen}
\label{snmp-tabellen}

Behalve scalaire waarden is het ook mogelijk om tabellen te gebruiken met SNMP.
We maken gebruik van de standaardtabel voor netwerkinterfaces \emph{ifTable} met als \gls{oid} 1.3.6.1.2.1.2.2 ter illustratie.
Een tabel wordt als volgt gedefiniëerd in een \gls{mib}-bestand: 
je hebt enerzijds een tabelobject (hier ifTable) en anderzijds een rijobject (hier ifEntry).
De definitie van het tabelobject zie je in \cref{definitie-iftable}.
De volgende codefragmenten tonen de definities van de overige objecten. %TODO: lstlistingname

Het tabelobject wordt gedefiniëerd als een sequentie of opeenvolging van rijobjecten.
Het rijobject wordt op zijn beurt gedefiniëerd als een sequentie van kolommen.
Hiervoor wordt een aparte sequentiedefinitie (IfEntry) gebruikt die 
bestaat uit een opeenvolging van kolomnamen gevolgd door hun datatype.

\begin{lstlisting}[language=asn.1, float=h, caption={Definitie van ifTable}, label=definitie-iftable]
ifTable OBJECT-TYPE
	SYNTAX	SEQUENCE OF IfEntry
	ACCESS	not-accessible
	STATUS	mandatory
	DESCRIPTION
			"A list of interface entries.  The number of
			entries is given by the value of ifNumber."
	::= { interfaces 2 }
\end{lstlisting}

\begin{lstlisting}[language=asn.1, float=h, caption={Sequentiedefinitie voor een tabelrij}, label=definitie-sequentie-rij]
IfEntry ::=
	SEQUENCE {
		ifIndex
			INTEGER,
		ifDescr
			DisplayString,
		ifType
			INTEGER,
		...
	}
\end{lstlisting}

Bij de definitie van het rijobject zelf zie je dat er als syntax (datatype) de voorgaande sequentiedefinitie wordt gebruikt.
Merk op dat de sequentiedefinitie (IfEntry) begint met een hoofdletter maar de definitie van het object zelf niet (ifEntry).

\begin{lstlisting}[language=asn.1, float=h, caption={Definitie van een rijobject}, label=definitie-rijobject]
ifEntry OBJECT-TYPE
	SYNTAX	IfEntry
	ACCESS	not-accessible
	STATUS	mandatory
	DESCRIPTION
			"An interface entry containing objects at the
			subnetwork layer and below for a particular
			interface."
	INDEX	{ ifIndex }
	::= { ifTable 1 }
\end{lstlisting}

Ook belangrijk is het INDEX-attribuut. Deze geeft aan hoe de tabel geïndexeerd moet worden, dus hoe je aan individuele rijen kunt geraken.
Dit kan een eenvoudige integer zijn of een samenstelling van kolommen die een rij uniek kan identificeren.
Hier wordt ifIndex gedefiniëerd als een eenvoudige integer.

\begin{lstlisting}[language=asn.1, float=h, caption={Definitie van ifIndex}, label=definitie-ifindex]
ifIndex OBJECT-TYPE
	SYNTAX	INTEGER
	ACCESS	read-only
	STATUS	mandatory
	DESCRIPTION
			"A unique value for each interface.  Its value
			ranges between 1 and the value of ifNumber.  The
			value for each interface must remain constant at
			least from one re-initialization of the entity's
			network management system to the next re-
			initialization."
	::= { ifEntry 1 }
\end{lstlisting}

De indexering gebeurt dan als volgt. Om te beginnen heb je de \gls{oid} van de tabel zelf: 1.3.6.1.2.1.2.2.
Het rijobject heeft ook een eigen \gls{oid}. Zijn volgnummer is 1 dus die komt achter de \gls{oid} van de tabel.
Daarna komt het volgnummer van de kolom. Laten we als voorbeeld de kolom met de snelheid van de interface nemen (ifSpeed).
Deze heeft als volgnummer 5. De \gls{oid} voor die kolom wordt dan 1.3.6.1.2.1.2.2.1.5.
Als we alle \glspl{oid} overlopen die beginnen met de \gls{oid} van de ifSpeed kolom, dan krijgen we de waarden van alle rijen voor die ene kolom.
Als we ook nog eens een specifieke rij willen opgeven dan volgt dat na de kolomaanduiding.
Vermits er hier gebruik gemaakt wordt van een simpele integer als index kunnen we achteraan de \gls{oid} 1 toevoegen om 
de snelheid van de \emph{eerste} interface te weten te komen. De uiteindelijke \gls{oid} wordt dan 1.3.6.1.2.1.2.2.1.5.1.
Je kunt dit visueel ook bevestigen in \cref{boomstructuur-tabel}.

Omdat de kolomaanduiding voor de rijaanduiding komt in een \gls{oid},
zal het overlopen van alle \glspl{oid} die beginnen met de \gls{oid} van de tabel tot resultaat hebben dat de tabel kolom per kolom wordt overlopen.
Deze operatie wordt een SNMP walk van een \gls{oid} genoemd en wordt later verder uitgelegd.

\begin{figure}[h]
	\centering
	\includegraphics[resolution=110]{figures/snmp/ifTable-cropped}
	\caption{Boomstructuur van een tabel}
	\label{boomstructuur-tabel}
\end{figure}


\section{SNMP Operaties}
\label{snmp-operaties}
SNMP maakt gebruik van het \gls{pdu} berichtformaat voor al het communicatieverkeer tussen SNMP-agents en \glspl{nms}.
Er worden een aantal verschillende operaties ondersteund door SNMP en die hebben elk hun eigen \gls{pdu}-formaat\cite{essentialsnmp}.
Hieronder worden enkel de belangrijkste SNMP-operaties gegeven die relevant zijn voor de masterproef:

\begin{itemize}
	\item GET
	\item GETNEXT
	\item GETBULK
\end{itemize}


\subsection{GET}
De GET-operatie was de eerste en belangrijkste SNMP operatie.
De GET-operatie wordt geïnitieerd door de \gls{nms} en laat die toe om een gegeven van een SNMP-agent op te vragen.
Om te weten welk gegeven de \gls{nms} nu juist wenst te weten te komen geeft die een \gls{oid} mee met de \gls{pdu} die het gegeven uniek zal identificeren.
Herinner je dat zowel de SNMP-agent als \gls{nms} over dezelfde \gls{mib} beschikken die dat \gls{oid} definiëert zodat ze beiden perfect weten waarover het gaat.
De SNMP-agent ontvangt de \gls{pdu} en verwerkt deze. Een nieuw GET-response \gls{pdu} wordt opgesteld voor dat \gls{oid} met de waarde ervan ingevuld en
terug verstuurd door de SNMP-agent. De \gls{nms} ontvangt de GET-response \gls{pdu} en weet nu de waarde voor dat \gls{oid}.

\todo[inline]{Figuur?}


\subsection{GETNEXT}
\label{snmp-getnext}
De GETNEXT-operatie wordt gebruikt om een verzameling van opeenvolgende \glspl{oid} op te vragen.
Gegeven een bepaalde \gls{oid} zal de GETNEXT-operatie de eerstvolgende \gls{oid} met bijhorende waarde teruggeven.
Dit gebeurt in lexicografische volgorde en vermits \glspl{oid} samengesteld zijn door integers kan zo makkelijk een ganse boomtak overlopen worden.
Deze manier van werken noemt men diepte-eerst zoeken.\cite{essentialsnmp}
Wanneer de \gls{nms} het antwoord ontvangt van een GETNEXT-operatie zal ze een nieuwe sturen voor het volgende \gls{oid}.
De \gls{nms} zal blijven GETNEXT-\glspl{pdu} sturen tot dat de agent een foutmelding terugstuurt die aangeeft dat het einde van de \gls{mib} bereikt is.

De GETNEXT-operatie is vooral handig voor het overlopen van tabellen.
Tabellen zijn niet noodzakelijk sequentieel geïndexeerd, ze kunnen bijvoorbeeld samengesteld zijn aan de hand van een aantal attributen.
Maar met de GETNEXT-operatie hoef je daar geen rekening mee te houden, de GETNEXT-operatie geeft je meteen de volgende geldige index met de bijhorende gegevens.
De GETNEXT-operatie uitgevoerd op de OID van een tabel geeft je de eerste geldige index.
Daarmee kun je dan alle volgende indexen opvragen.

Normaal gezien zal je niet rechtstreeks in contact komen GETNEXT-operaties, maar zul je gebruik maken van de SNMP walkopdracht.
Je hoeft enkel de OID mee te geven met de SNMP walkopdracht waarop deze de nodige GETNEXT-requests zal sturen om de ganse boomtak te overlopen.
\todo{Herhaalt eigenlijk wat er hierboven werd gezegd, maar ipv NMS -> SNMP Walk}

\todo[inline]{Figuur?}


\subsection{GETBULK}
Met de tweede versie van SNMP werd de GETBULK-operatie gespecificeerd.
Deze operatie laat je toe om in een request meteen een heleboel \glspl{oid} op te vragen.
Bij een gewone GET-operatie kun je ook meerdere \glspl{oid} meegeven maar de berichtgrootte wordt beperkt door de capaciteit van de SNMP-agent.
Als een SNMP-agent geen antwoord kan geven op alle \glspl{oid} die werden gevraagd in de GET-operatie wordt een foutboodschap teruggestuurd zonder data.
De GETBULK-operatie daarentegen probeert zo veel mogelijk data terug te sturen als het kan.
Met de GETBULK-operatie is het dus wel mogelijk om incomplete antwoorden terug te krijgen.\cite{essentialsnmp}

\section{Manieren om SNMP gegevens op te vragen}
\label{manieren-om-snmp-gegevens-op-te-vragen}

Aan de hand van de SNMP operaties besproken in \cref{snmp-operaties} zijn er verschillende mogelijkheden om gegevens op te vragen
al naar gelang of je een of meerdere gegevens wenst op te vragen en welke gegevens je juist wenst op te vragen (bv. tabellen).
Met behulp van Net-SNMP zullen deze verschillende mogelijkheden hieronder gedemonstreerd worden.
Net-SNMP is een gratis en open-source softwarepakket die commando's aanbiedt voor de verschillende SNMP-operaties en
is beschikbaar voor de meeste besturingssystemen waaronder Windows en Linux.
Wij maken gebruik van de Linux versie.


\subsection{Een gegeven}
%TODO: Goede titel? Enkelvoudig/meervoudig? Een/meerdere gegevens?

\subsubsection{GET-operatie}

Indien je maar geïnteresseerd bent in slechts een gegeven dan ligt het voor de hand om gebruik te maken van de SNMP GET-operatie.
Voorwaarde is wel dat je exact de \gls{oid} kent van het gegeven dat je wenst op te vragen.
Als we gebruik maken van het gratis softwarepakket Net-SNMP dat zowel voor de meeste besturingssystemen beschikbaar is,
dan ziet een GET-request er ongeveer zo uit:

\begin{lstlisting}[float=h, caption={SNMP GET-opdracht}, label=netsnmp-get]
$ snmpget -v 1 -c public 127.0.0.1 sysDescr.0
SNMPv2-MIB::sysDescr.0 = STRING: Linux debian-vm-01 3.2.0-4-amd64 #1 SMP Debian 3.2.54-2 x86_64
\end{lstlisting}

Daarbij geeft de -v optie de te gebruiken versie mee en de -c optie de communitystring.
De opties worden gevolgd door het IP adres van het te ondervragen toestel en het \gls{oid} van het gegeven dat je wenst op te vragen.
Zoals je misschien al opgemerkt hebt, wordt de \gls{oid} niet voorafgegaan door iso.org.dod.internet.mgmt.mib-2.system zoals het zou horen
voor een \gls{oid} die in die subtak valt. De reden daarvoor is dat dit niet vereist is als de naam van de tak uniek is, wat het geval is voor sysDescr.

De volgende vraag die je wellicht stelt is waarom er nog een .0 achter de \gls{oid} staat.
Dit is omdat MIB objecten geïdentificeerd worden door de conventie x.y waarbij x de \gls{oid} is van het object
en y de instantie aanduidt. Normaal wordt y gebruikt bij tabellen om de rij aan te duiden (1 is de eerste rij, 2 de tweede, enzovoort) zoals
uitgelegd in \cref{snmp-tabellen}.
Maar bij scalaire objecten is y altijd 0. De .0 achterwege laten levert een fout op.\cite{essentialsnmp}

\subsubsection{GETNEXT-operatie}

Je kan ook gebruik maken van de SNMP GETNEXT-operatie als je het \emph{volgende} gegeven wenst te weten te komen.
Dit wordt hoofdzakelijk gebruikt bij het overlopen van tabellen zonder dat je moet rekening houden met de indexering van de tabel
die immers niet noodzakelijk sequentieel is.
Het is mogelijk om zelf een SNMP GETNEXT-operatie uit te voeren, maar normaliter maak je gebruik van de SNMP walkopdracht die
hiervan gebruik maakt om een ganse boomtak te overlopen (zie \cref{snmp-getnext}).
Als je de indexering van een tabel niet kent, dan zou je het volgende kunnen doen:

\begin{lstlisting}[float=h, caption={SNMP GETNEXT-opdracht op een tabel}, label=netsnmp-getnext]
$ snmpgetnext -v 1 -c public 127.0.0.1 ifTable
IF-MIB::ifIndex.1 = INTEGER: 1
\end{lstlisting}

Dit levert dan de waarde van de eerste kolom (hier ifIndex) in de eerste rij op.
Herinner je dat de \gls{oid} van een tabelelement eerst het kolomnummer bevat en dan de index van de rij.
Dus als je de waarde van een andere kolom wenst zonder dat je de index van de eerste rij kent zou je het volgende kunnen doen:

\begin{lstlisting}[float=h, caption={SNMP GETNEXT-opdracht op een kolom van een tabel}, label=netsnmp-getnextcol]
$ snmpgetnext -v 1 -c public 127.0.0.1 ifTable.1.2
IF-MIB::ifDescr.1 = STRING: lo
\end{lstlisting}

Dit levert dan de tweede kolom (de beschrijving van de interface, hier lo, kort voor loopback interface) op van de eerste rij.
Maar zoals gezegd zul je eerder gebruik maken van de SNMP walkopdracht dan GETNEXT-operaties (zie verder).


\subsection{Meerdere gegevens}

\subsubsection{SNMP BULK-operatie}
Als je meerdere gegevens wenst op te vragen gebruik je best de SNMP BULK-operatie.
Deze combineert meerdere \glspl{oid} in een bericht waardoor je sneller meerdere gegevens kunt opvragen.
In feite zal de SNMP BULK-operatie simpelweg meerdere GETNEXT-operaties uitvoeren en de resultaten combineren in een bericht.

Om van SNMP BULK-operaties gebruik te maken moet je ten eerste zorgen dat je gebruik maakt van SNMP versie 2c (zie ook \cref{snmp-versies}).
Ten tweede moet je ook twee extra parameters opgeven: \textit{non-repeaters} en \textit{max-repetitions}.
Je kunt meerdere \glspl{oid} meegeven met de BULK-operatie en met die parameters kun je opgeven of er slechts een keer
een GETNEXT-operatie op een \gls{oid} moet uitgevoerd worden of meerdere.
Non-repeaters geeft aan hoeveel \glspl{oid} er zijn meegegeven waarop slechts een keer een GETNEXT-operatie moet uitgevoerd worden.
Deze \glspl{oid} zullen dus elk slechts een antwoord bevatten in het antwoordbericht.
Met max-repetitions geef je aan hoeveel keer er een GETNEXT-operatie moet uitgevoerd worden op de overige \glspl{oid}.
Als je max-repetitions op 10 zet, dan zal je voor elk van de overige \glspl{oid} 10 antwoorden krijgen in het antwoordbericht.
Vermits je enkel opgeeft hoeveel \glspl{oid} niet herhalen (non-repeaters) moet je eerste de niet-herhalende \glspl{oid} opgeven en dan de herhalende.
Wellicht dat een voorbeeld meer duidelijkheid verschaft:

\begin{lstlisting}[float=h, caption={SNMP BULK-opdracht}, label=netsnmp-bulk]
$ snmpbulkget -v 2c -c public -Cn1 -Cr3 127.0.0.1 sysDescr ifTable ipAddrTable
SNMPv2-MIB::sysDescr.0 = STRING: Linux debian-vm-01 3.2.0-4-amd64 #1 SMP Debian 3.2.54-2 x86_64
IF-MIB::ifIndex.1 = INTEGER: 1
IP-MIB::ipAdEntAddr.10.0.2.11 = IpAddress: 10.0.2.11
IF-MIB::ifIndex.2 = INTEGER: 2
IP-MIB::ipAdEntAddr.127.0.0.1 = IpAddress: 127.0.0.1
IF-MIB::ifIndex.3 = INTEGER: 3
IP-MIB::ipAdEntIfIndex.10.0.2.11 = INTEGER: 2
\end{lstlisting}

Het aantal non-repeaters en het aantal max-repetitions worden ingesteld met respectievelijk de -Cn en de -Cr optie.
Vermits het aantal non-repeaters op 1 staat wil dit zeggen dat enkel de eerste \gls{oid}, sysDescr, niet herhaald moet worden.
En omdat max-repetitions op 3 staat, worden alle overige \glspl{oid} 3 keer herhaald (hier ifTable en ipAddrTable).
Als antwoord zie je het volgende staan: de waarde van sysDescr, de eerste 3 waarden van ifTable en de eerste 3 waarden van ipAddrTable.

Merk op dat de versie nu op 2c staat. Verder zie je dat de \gls{oid} voor de scalaire waarde sysDescr nu niet gevolgd wordt door .0,
want achterliggend maakt de BULK-operatie gebruik van een GETNEXT-operatie.
Als je toch .0 toevoegt dan zul je niet de waarde van sysDescr terugkrijgen maar die van de volgende \gls{oid}.
Moest je ten slotte een hoger aantal max-repetitions gebruiken dan een tabel cellen heeft,
of een scalaire waardere meerdere keren laten herhalen, dan krijg je ook nog de waarden van de \glspl{oid} die volgen op de tabel of de scalaire waarde terug.


\subsubsection{GET- en GETNEXT-opdrachten}

In principe is het ook mogelijk om meerdere gegevens op te vragen met een GET- of GETNEXT-opdracht.
Om dat te doen geef je simpelweg meerdere \glspl{oid} mee in de opdracht:

\begin{lstlisting}[float=h, caption={Meerdere gegevens opvragen met SNMP GET}, label=netsnmp-get-meerdere]
$ snmpget -v 1 -c public 127.0.0.1 sysDescr.0 sysUpTime.0 sysName.0
SNMPv2-MIB::sysDescr.0 = STRING: Linux debian-vm-01 3.2.0-4-amd64 #1 SMP Debian 3.2.54-2 x86_64
DISMAN-EVENT-MIB::sysUpTimeInstance = Timeticks: (1451759) 4:01:57.59
SNMPv2-MIB::sysName.0 = STRING: debian-vm-01
\end{lstlisting}

Het nadeel van dit te doen met een GET- of GETNEXT-opdracht is dat, als er een fout optreedt bij het opvragen
van een van de \glspl{oid}, je geen gegevens terug zal krijgen over de \glspl{oid} die wel correct waren:

\begin{lstlisting}[float=h, caption={Meerdere gegevens opvragen met SNMP GET met een foute OID}, label=netsnmp-get-meerdere-fout]
$ snmpget -v 1 -c public 127.0.0.1 sysDescr.0 sysUpTime.0 sysName.0 fouteOID
fouteOID: Unknown Object Identifier (Sub-id not found: (top) -> fouteOID)
\end{lstlisting}

Bij een BULK-opdracht krijg je die gegevens wel.
Om meerdere gegevens op te vragen maak je dus beter gebruik van BULK-opdrachten.


\subsubsection{SNMP walk}

Zoals gezegd maak je normaal gezien weinig gebruik van de GETNEXT-operatie maar gebruik je in de plaats de SNMP walkopdracht.
De SNMP walkopdracht wordt gebruikt om een ganse boomtak te overlopen en maakt achterliggend gebruik van de GETNEXT-operaties.
Met de volgende SNMP walkopdracht kun je de interfacetabel van een toestel overlopen:

\begin{lstlisting}[caption={SNMP walkopdracht}, label=netsnmp-walk]
$ snmpwalk -v 1 -c public 127.0.0.1 ifTable
IF-MIB::ifIndex.1 = INTEGER: 1
IF-MIB::ifIndex.2 = INTEGER: 2
IF-MIB::ifIndex.3 = INTEGER: 3
...
IF-MIB::ifSpecific.4 = OID: SNMPv2-SMI::zeroDotZero
IF-MIB::ifSpecific.5 = OID: SNMPv2-SMI::zeroDotZero
IF-MIB::ifSpecific.6 = OID: SNMPv2-SMI::zeroDotZero
\end{lstlisting}

Merk op dat we hier geen instantie (de .0 bij scalaire gegevens) na de \gls{oid} hebben opgegeven!
Moest je hier toch een instantie opgeven (de index van een rij van de tabel) dan zou je enkel die ene rij terugkrijgen
en had je hetzelfde kunnen bereiken met een SNMP GET-opdracht.

Vermits de BULK-operaties ook gebruik maken van GETNEXT-operaties maar die bundelen in een antwoord,
kun je de SNMP walkopdracht nog sneller laten verlopen door de BULK variant te gebruiken.
Het resultaat blijft natuurlijk hetzelfde.

\begin{lstlisting}[caption={SNMP walkopdracht m.b.v. BULK-operaties}, label=netsnmp-bulkwalk]
$ snmpbulkwalk -v 2c -c public 127.0.0.1 ifTable
...
\end{lstlisting}

Let erop dat je de versie moet veranderen omdat je gebruik maakt van BULK-operaties!

Net als bij het snmpbulkget commando kun je hier echter nog het aantal herhalingen (max-repetitions) opgeven met de -Cr optie.
Daarbij komt het erop neer dat we kunnen opgeven hoeveel objecten moeten samengebundeld worden in een pakket.
Om een SNMP walk zo snel mogelijk te laten verlopen zijn we natuurlijk geneigd om dit aantal zo hoog mogelijk in te stellen.
We moeten wel rekening houden met twee beperkingen: de maximale grootte van een pakket is beperkt tot de \gls{mtu} van een verbinding.
Bij een ethernetverbinding gaat het om 1500 Bytes.

Bij een groot aantal herhalingen is het ook mogelijk dat er redelijk wat data verspild wordt.
Stel dat we het aantal herhalingen instellen op 50, dan zullen de pakketten steeds 50 objecten bevatten.
Ook het laatste pakket zal er 50 bevatten, ook al valt er slechts een object meer binnen de walk.
Dan zijn de overige 49 objecten natuurlijk niet relevant voor ons en doen we daar dan ook niks mee.

Als we een walk doen van een deelboom die maar 10 objecten heeft, heeft het ook geen zin om het aantal herhalingen op 50 te zetten.
Natuurlijk weet je op voorhand niet hoeveel objecten juist onder een \gls{oid} vallen,
maar je kan meestal wel een schatting doen of dit baseren op het aantal objecten dat in het verleden werd opgehaald voor dat \gls{oid} en voor dat toestel.

Het aantal herhalingen is uiteindelijk een beslissing die vooral zal afhangen van het geschatte aantal gegevens dat opgevraagd gaat worden.
Standaard staat het aantal herhalingen daarom op een conservatieve 10 objecten.


\section{Versies}
\label{snmp-versies}
Er zijn drie grote versies van SNMP die momenteel in gebruik zijn op netwerktoestellen: SNMPv1, SNMPv2c en SNMPv3.
Ondanks dat SNMPv3 de twee voorgaande versies opvolgt kun je nog steeds veel toestellen vinden die enkel met SNMPv2c of zelfs met SNMPv1 werken.
Opnieuw is er een afhankelijkheid van de fabrikant om voor de ondersteuning van SNMPv3 te zorgen.


\subsection{SNMPv1}
De eerste versie van SNMP dateert al van 1988 maar is soms toch nog te vinden als enige ondersteunde versie op een toestel.
Een van de grootste gebreken die deze versie kenmerkt is de beveiliging.
De originele versie van SNMP maakte gebruik van een zogeheten \emph{community string} om SNMP-requests te beveiligen.
Die string fungeerde in feite als een soort wachtwoord die in cleartext met ieder SNMP-request werd meegegeven\cite{snmp-wiki}.
Natuurlijk kan iedereen die de requests kan onderscheppen het wachtwoord gewoon uitlezen en zelf requests met dat wachtwoord versturen.
SNMP voorziet in de mogelijkheid om niet alleen data uit te lezen, maar ook om data aan te passen met behulp van de SNMP SET-operatie.
Vanwege de slechte beveiliging wordt de SET-operatie echter zo goed als nooit geïmplementeerd in de SNMP-agent zodat de operatie geen effect heeft.

\todo[inline]{Waarom werd deze versie geaccepteerd? Werd aanzien als tijdelijke oplossing.}


\subsection{SNMPv2c}
De tweede versie van SNMP introduceerde in 1993 \cite{snmp-versions} oorspronkelijk de GETBULK-operatie en een betere beveiliging.
De nieuwe beveiliging werd echter als te complex beschouwd en werd op veel plaatsen niet aanvaard.
SNMPv2c werd in 1996 \cite{snmp-versions} voorgesteld als een alternatief die de verbeteringen van SNMPv2 had maar de complexe beveiliging ervan achterwege liet
ten voordele van de community strings van SNMPv1.
Deze versie werd wel aanvaard door de gemeenschap en is nog steeds in gebruik op een groot aantal toestellen.
De GETBULK-operatie was een welkome toevoeging aan SNMP omdat ze een veel performanter alternatief bood voor de vele GETNEXT-operaties
die voorheen nodig waren om grote hoeveelheden gegevens op te vragen.

\todo[inline]{Cross-compatibility met vorige versie}


\subsection{SNMPv3}
De derde en laatste versie van SNMP werd geaccepteerd als een volwaardige internetstandaard in 2002 \cite{snmpv3} en
bracht de reeds lang gevraagde verbetering in beveiliging met zich mee.
Omdat de vorige versies van SNMP slecht beveiligd waren werd het protocol enkel gebruikt voor het monitoren van het netwerk en performantiebeheer.
Met de verbeterde beveilging in SNMPv3 kan SNMP eindelijk een veilig platform aanbieden om niet enkel je netwerk passief te beheren zoals voorheen,
maar ook actief te beheren door configuratiewijzigingen uit te voeren via SNMP.
SNMPv3 is echter op veel toestellen nog steeds niet aanwezig en zeker de SNMP SET-operatie wordt in de praktijk maar zelden geïmplementeerd.

\section{Berichtstructuur van SNMP}
\label{snmp-berichtstructuur}

In deze paragraaf bespreken we de berichtstructuur van een SNMP-bericht.
In tegenstelling tot bijvoorbeeld IP- en ethernetheaders, hebben de headers van SNMP-berichten geen vaste lengte.
In de plaats daarvan bestaat een SNMP-bericht uit Tag-Length-Value (TLV) tripletten\cite{moreau}.
Daarbij geeft de tag aan om wat voor datatype het gaat, de length duidt de lengte aan van de data en value bevat de data zelf.
Je hebt primitieve datatypes zoals een integer, een octetstring of een \gls{oid}.
Maar je hebt ook complexe datatypes zoals een sequentie of een \gls{pdu} voor bijvoorbeeld een GET-request of een GET-response.
De complexe types zoals de sequentie zijn opgebouwd uit meerdere kleinere velden zodat je een geneste structuur krijgt\cite{snmp-message-format}.

\begin{table}[h]
\centering
\begin{tabular}{@{}ll@{}}
\toprule
Primitieve datatypes & Complexe datatypes \\ \midrule
Integer              & Sequence           \\
Octet String         & GetRequest         \\
Object Identifier    & GetResponse        \\
Null                 & GetBulkRequest     \\ \bottomrule
\end{tabular}
\caption{Enkele SNMP datatypes}
\label{tabel-datatypes}
\end{table}

Een SNMP-bericht is dus niks meer dan een geneste structuur van datavelden.
Het SNMP-bericht zelf wordt gedefinieerd als een sequentie van drie velden: een integer die de SNMP versie voorstelt, een octetstring die de community voorstelt en
de eigenlijke SNMP-\gls{pdu} (zelf een samengesteld type).

\begin{figure}[h]
	\centering
	\includegraphics[scale=0.45]{figures/snmp/berichtstructuur-1}
	\caption{Berichtstructuur van een SNMP-bericht}
	\label{fig-berichtstructuur-1}
\end{figure}

De SNMP-\gls{pdu} bestaat uit de volgende velden:

\begin{itemize}
	\item Request ID (integer): unieke identificatie van een SNMP-request.
		Het responsebericht zal dezelfde identificatie gebruiken zodat de ontvanger weer bij welke request het antwoord hoort.
	\item Error (integer): duidt aan of er een fout is opgetreden.
		De SNMP-agent zal deze waarde veranderen indien nodig. Als de waarde op nul staat is er geen fout opgetreden.
	\item Error Index (integer): verwijst naar het object dat de fout heeft veroorzaakt.
	\item Varbind List (sequentie): de verzameling van alle objecten in de \gls{pdu}.
		\begin{itemize}
			\item Varbind (sequentie): komt overeen met een object en bevat zijn \gls{oid} en zijn waarde (enkel ingevuld in het antwoordbericht).
				\begin{itemize}
					\item Object Identifier (OID)
					\item Value (integer, octetstring, \ldots)
				\end{itemize}
		\end{itemize}
\end{itemize}

\begin{figure}[h]
	\centering
	\includegraphics[scale=0.45]{figures/snmp/berichtstructuur-2}
	\caption{Berichtstructuur van een SNMP-bericht en zijn SNMP-\gls{pdu}}
	\label{fig-berichtstructuur-2}
\end{figure}

Een voorbeeld van een GET-request in detail zie je in \cref{fig-berichtstructuur-3}.
Onderaan zie je de hexadecimale waarden van de bytes.
De eerste byte van een veld duidt steeds de code van het datatype aan en de tweede de lengte van het veld.
De volgende bytes vormen dan de inhoud van het veld.
Zodoende krijgen we de Tag-Length-Value tripletten waar we over spraken.
Merk op dat als de lengte van een veld verandert dat deel uitmaakt van een complex datatype zoals een sequentie,
dan moet de lengte van het bovenliggende datatype ook aangepast worden.

\begin{figure}[h]
	\centering
	\includegraphics[scale=0.40]{figures/snmp/berichtstructuur-3}
	\caption{Berichtstructuur van een SNMP-bericht in detail}
	\label{fig-berichtstructuur-3}
\end{figure}

\todo[inline]{Bronvermelding figuren.}

% Bestaande situatie
\chapter{Bestaande situatie}

\todo[inline]{Herbekijk (zie ook comments)}

\section{SNMP Data Retriever}
\label{snmp-data-retriever}
De SNMP Data Retriever is het stuk software dat NetworkMining zelf ontwikkeld heeft voor de bevraging van netwerkcomponenten via SNMP.
Alhoewel het de bedoeling is om netwerkcomponenten te ondervragen over bijvoorbeeld hun routetabel of \gls{arptabel},
is het geen probleem om ook alle andere soorten toestellen te ondervragen, bijvoorbeeld werkstations van werknemers of
telefoontoestellen die verbonden zijn met het netwerk. De retriever laat je toe om een lijst van op te vragen toestellen op te geven.
Om te weten welke gegevens moeten opgevraagd worden bij elk toestel, moet aan elk toestel een toesteltype toegewezen worden.
De toesteltypes maak je zelf aan en bevatten een lijst van de op te vragen gegevens voor dat type.
Hieronder geven we een overzicht van de verschillende configuratiemogelijkheden die de \nwmretriever{} aanbiedt.

We bespreken ook de databankstructuur die door de \nwmretriever{} gebruikt wordt om de resultaten weg te schrijven
en geven ook een globaal overzicht van hoe de software te werk gaat.

\todo[inline]{Relevant om dit hier nog eens te herhalen? V}
Bij de aanvang van de masterproef was het zo dat de retriever reeds ingezet werd voor kleine en middelgrote netwerken.
De bedoeling van de masterproef is om de nodige aanpassingen te doen zodat de retriever ook vlot kan ingezet worden voor grote netwerken,
van 1000 toestellen en meer.
% Het plan is om eind dit jaar de software ook te kunnen gebruiken voor het netwerk van Telenet. \todo{Mag Telenet vermeld worden?}

\todo[inline]{Samenvatting van de werking van de retriever? GET- \& GETNEXT-requests, wegschrijven van de resultaten in een DB}

\todo[inline,caption=bestaande-snmp-retriever]{Uitleg over bestaande SNMP Data Retriever. \\
Vandaag in gebruik bij NetworkMining voor kleine netwerken.
Bedoeling om te kunnen inzetten op grote netwerken zoals dat van Telenet. \\
Beschrijving huidige functionaliteit, hoe werkt de software, configuratie zie hieronder.}

\subsection{Configuratie}
\label{snmp-data-retriever-configuratie}
Er zijn drie manieren waarop de retriever kan geconfigureerd worden. De belangrijkste,
en wellicht de enige waar de eindgebruiker mee in aanraking zal komen is het XML-configuratiebestand.
Een aantal opties kunnen ook doorgeven worden aan de hand van argumenten bij het oproepen van het programma.
De laatste manier is via nog een configuratiebestand: het \emph{AppConfig}-bestand. De laatste twee mogelijkheden
zullen hoogstwaarschijnlijk eenmalig geconfigureerd worden bij de installatie van de software en verder nooit meer gewijzigd worden.

% We bespreken de configuratiemogelijkheden die aangeboden worden door elke optie en hoe die ingesteld moeten worden.

\todo[inline]{3 manieren om de retriever te configureren, zie de puntjes hieronder.}

\subsubsection{XML-configuratiebestand}

De twee belangrijkste dingen die je kunt configureren in het XML-configuratiebestand, zijn wat voor types toestellen er zijn en de lijst van IP-adressen van die toestellen.
Een voorbeeld van een definitie van een toesteltype zie je in \cref{xmlconfig-typedefinition}.
Het toesteltype heet \emph{General} en als er een toestel van dat type bevraagd wordt,
moet er een SNMP walk gedaan worden van de \textit{system} boomtak met als \gls{oid} 1.3.6.1.2.1.1.
Dit is dezelfde boomtak die we eerder hebben gezien in \cref{boomstructuur}.
De system boomtak is verplicht aanwezig op alle SNMP-toestellen, vandaar de naam van het toesteltype.
Het \gls{mib}-bestand waarin die tak gedefiniëerd is heet \emph{RFC1213-MIB}.
Deze gegevens vind je terug als de attributen van de SNMP walk-opdracht in het configuratiebestand.
Een toesteltype moet niet beperkt zijn tot een enkele SNMP walk. Ze kan ook bestaan uit meerdere SNMP walk-opdrachten,
of zelfs een SNMP GET-opdracht voor een enkele \gls{oid}.

% Make a float of the code listing so it doesn't get broken up in a page break.
\begin{lstlisting}[language=XML, float=h, caption={Definitie van een toesteltype in het XML-configuratiebestand}, label=xmlconfig-typedefinition]
<deviceType name="General">
	<snmpWalk oid="1.3.6.1.2.1.1" mib="RFC1213-MIB" name="system" />
</deviceType>
\end{lstlisting}

Nadat de toesteltypes gedefiniëerd zijn, kun je de IP-adressen opgeven van alle toestellen die opgevraagd moeten worden.
Bij de opsomming van de toestellen hoort natuurlijk ook hun toesteltype die we eerder gedefiniëerd hebben.
In \cref{xmlconfig-devicedefinition} zie je dat de definitie van een toestel bestaat uit drie attributen: een arbitrair gekozen naam, zijn IP-adres of hostnaam en zijn toesteltype. 

\begin{lstlisting}[language=XML, float=h, caption={Definitie van een toestel in het XML-configuratiebestand}, label=xmlconfig-devicedefinition]
<device name="atlas2a1.intec.ugent.be" ip="atlas2a1.intec.ugent.be" type="Bridge" />
\end{lstlisting}

De overige opties zijn die voor een databaseconnectiestring, de communitystring, de locatie van de \gls{mib}-bestanden en de SNMP-timeout waarde.
Dit is hoelang gewacht wordt (in milliseconden) op een response na het versturen van een SNMP-request.

\begin{lstlisting}[language=XML, float=h, caption={Overige opties in het XML-configuratiebestand}, label=xmlconfig-misc]
<database value="Database=snmpdb;Data Source=localhost;User Id=networkminer;Password=SomePassword;Port=3306;old syntax=yes" />
<snmpCommunity get="public" />
<MIBpath value=".\MIBs" />
<snmpTimeout value="3000" />
\end{lstlisting}


\subsubsection{Argumenten}
Bij het oproepen van de retriever kun je optioneel enkele argumenten meegeven.
De belangrijkste twee zijn \emph{inputfile} en \emph{clearresults}.
Met het eerste argument geef je de locatie mee van het XML-configuratiebestand.
Clearresults zorgt ervoor dat resultaten van een vorige retrieval gewist worden, zodat je met een schone lei begint.
In \cref{retriever-argumenten} zie je een voorbeeld van hoe je deze argumenten moet gebruiken.

\begin{lstlisting}[float=h, caption={Oproepen van SNMP Data Retriever met twee argumenten}, label=retriever-argumenten]
SNMPDataRetrieval.exe "-clearresults" "-inputfile=config\snmp.xml"
\end{lstlisting}


\todo[inline, caption={Argumenten van retriever}]{
Zie ParseCommandLineAttributes():

\begin{itemize}
	\item clearresults
	
	\item noretrieve
	
	\item getdevicesfromquery
	
	\item serial
	
	\item inputfile
\end{itemize}
}

\subsubsection{AppConfig}
Met het AppConfig-bestand zal de eindgebruiker normaal niet in contact komen.
Het bevat de mogelijkheid om het loglevel te veranderen zodat meer of minder logging informatie wordt weggeschreven in het logbestand.
De databaseconfiguratiestring kan hier ook opgegeven worden,
maar, indien er in het XML-configuratiebestand ook een databank opgegeven wordt, dan wordt de laatste gebruikt.
Na het vervangen van het loggingframework (zie \cref{profiling}) komen er enkele loggingopties bij in dit bestand.
Zo kun je naast het loglevel ook extra uitvoermogelijkheden opgeven: naar een tekstbestand, naar het consolescherm, naar een databank of een combinatie.
Ook het logformaat kan hier desgewenst aangepast worden.

\subsection{Databankstructuur}
\label{snmp-data-retriever-db}
Tijdens het uitvoeren van de \nwmretriever{} worden er een drietal tabellen aangemaakt: \textit{types}, \textit{devices} en \textit{results}.
In de eerste tabel worden de toesteltypes opgeslagen.
Een voorbeeld zie je in \cref{fig-db-types}.
Daar werden twee toesteltypes gedefinieerd: bridge en router.
Voor elk toesteltype zijn er een aantal opdrachten (GET- of walkopdrachten), samen met hun datatype, \gls{oid}, \gls{mib}-bestand en naam.

\begin{figure}[h]
	\centering
	\includegraphics[scale=0.50]{figures/database/types}
	\caption{Tabel van toesteltypes en hun opdrachten}
	\label{fig-db-types}
\end{figure}

De toestellen zelf worden opgeslagen in de devices tabel.
In \cref{fig-db-devices} zie je een toestel met zijn IP-adres of hostnaam, het toesteltype en zijn communitystring.

\begin{figure}[h]
	\centering
	\includegraphics[scale=0.40]{figures/database/devices}
	\caption{Tabel van toestellen}
	\label{fig-db-devices}
\end{figure}

De laatste tabel bevat de opgehaalde gegevens.
Een aantal resultaten zie je in \cref{fig-db-results}.
Elk resultaat bevat de host, de \gls{oid}, het attribuutnaam (dat wordt samengesteld uit de naam van de \gls{mib} en de naam van de \gls{oid}),
de waarde van het resultaat en de datum waarop ze werd opgehaald.

\begin{figure}[h]
	\centering
	\includegraphics[scale=0.50]{figures/database/results}
	\caption{Tabel van resultaten}
	\label{fig-db-results}
\end{figure}


\subsection{Werking}
\label{werking}

In deze paragraaf geven we een \textit{high-level} overzicht van de werking van de \nwmretriever{}.
Bij het analyseren van de applicatie met een profiler wordt hier ook naar verwezen om te weten wat elke methode juist doet.

De \nwmretriever{} begint eerst en vooral met de \textit{Initialize}-methode op te roepen.
Deze is verantwoordelijk voor de configuratie en het opzetten van de databankverbinding.
Zowel de commandlineparameters als het XML-configuratiebestand worden hier verwerkt.
Na het inlezen van de configuratie wordt de databankverbinding opgezet met de gegevens uit de configuratie.

Nadat de basisconfiguratie werd ingelezen door de Initialize-methode worden de toesteltypes (devicetypes) ingelezen uit het XML-configuratiebestand.
Indien nodig wordt er in de databank een nieuwe devicetypestabel aangemaakt en worden de toesteltypes daarin opgeslagen.
Hetzelfde gebeurt voor de te bevragen toestellen.
Ten slotte wordt ook de results tabel aangemaakt, indien die nog niet bestond, om de opgehaalde gegevens in weg te schrijven.

Daarna kan het ophalen van de gegevens beginnen.
De lijst van de te bevragen toestellen wordt overlopen, en, om meerdere toestellen tegelijkertijd op te vragen,
wordt voor elk toestel een nieuwe thread gestart.
Het ophalen zelf gebeurt in de \textit{RetrieveFromDevice}-methode en elke thread voert die methode uit voor een toestel.

Er worden maximaal 50 threads aangemaakt.
Bijgevolg kunnen er maar 50 toestellen tegelijkertijd opgevraagd worden.
Eens er 50 threads aangemaakt zijn, wordt er gewacht tot een vorige thread klaar is om een nieuwe te starten.
De threads worden in een lijst bijgehouden.
In een oude versie van de \nwmretriever{} werd, na het starten van de eerste 50 threads, gewacht op
de eerste thread om nieuwe threads aan te maken.

De RetrieveFromDevice-methode overloopt de lijst van instructies die moeten afgehandeld worden voor een bepaald toestel(type).
Dat kunnen ofwel enkelvoudige gegevens zijn, die opgehaald moeten worden met een GET-request, ofwel moet een SNMP walk gedaan worden van een bepaalde \gls{oid}.
Vervolgens worden de request(s) gestuurd naar dat toestel.
Nadat het antwoord is ontvangen, worden alle variabelen die in dat antwoord zitten overlopen.
Er wordt gecontroleerd of de ontvangen gegevens relevant zijn (als ze onder de originele \gls{oid} vallen bij een SNMP walk),
en zoja wordt het gegeven weggeschreven in de databank met de \textit{InsertResultRow}-methode.
De InsertResultRow-methode doet niks meer dan het uivoeren van een SQL insert-query.
Gegevens worden dus een per een weggeschreven in de databank.


% Benchmarks en experimenten
\chapter{Benchmarks en Experimenten}
% \chaptermark{Een kortere titel voor de paginahoofding (verschillend van titel in TOC)}

\todo[inline,caption={}]{Terminologie?

\begin{itemize}
	\item Object (geretourneerde gegevens.) <-> OID's (kunnen subtakken zijn van meerdere OID's)
	\item SNMP Data Retriever
	\item Relevante OID's/objecten/gegevens
\end{itemize}

}

\section{Kleinschalige benchmarks en experimenten}

\todo[inline]{Schrijf een inleiding. \\
Enerzijds de Virtual Box opstelling. Anderzijds heb je ook de iMinds switchen. Ten slotte de profiler.}

\todo[inline]{Eigenlijk maakt het weinig uit om wat voor soort toestellen het gaat:
alhoewel de applicatie voornamelijk bedoeld is voor netwerkapparatuur, voor onze testen hebben we enkel een noodzaak aan voldoende gegevens die via SNMP kunnen opgevraagd worden.}

\subsection{VirtualBox}

\todo[inline]{Betere titel! Virtuele testopstelling van een handvol switches.}

\todo[inline, caption={}]{
Bespreking originele testopstelling van Wouter Tavernier? \\
Specs \\
Software(config)

\begin{itemize}
	\item screen (niet geinstalleerd)
	\item sudo
	\item snmpd
	\item snmp
	\item snmp-mibs-downloader
	\item bridge, lldp...
\end{itemize}


}

De eerste testopstelling bestaat uit vier virtuele machines die switches in een netwerk nabootsen.
Als virtualisatieplatform wordt er gebruik gemaakt van \textit{Oracle VM VirtualBox} (verder gewoon VirtualBox genoemd).
VirtualBox is vrij te verkrijgen voor alle gangbare besturingssystemen en is bovendien open-source.

\subsubsection{Hardwareconfiguratie}

Het gastheerbesturingssysteem is een 64-bit versie van Windows 7.
Windows en VirtualBox zijn geïnstalleerd op een SSD, maar de virtuele machines zelf staan op een magnetische harde schijf.

Aan elke node wordt 256 MB geheugen en een CPU kern (van een Intel Core i7 3610QM processor) toegewezen.
Op de nodes wordt een minimale versie van Debian 7 geïnstalleerd, zonder grafische schil.
Hierdoor is zelfs 256 MB een ruime luxe voor de nodes: na het opstarten van een node wordt er amper 70MB geheugen gebruikt.

Alle toestellen zijn rechtstreeks met elkaar verbonden in een privénetwerk.
Via \gls{nat} kunnen ze via de gastheer toch nog het internet bereiken.
Dit werd bewerkstelligd met de \textit{NAT Network mode},
een nieuwe feature in VirtualBox die nog niet in de documentatie staat maar wel kort beschreven wordt in een nieuwspost (zie\cite{vbox-nat-network-mode}).
Ter vergelijking:
\textit{NAT mode} laat gastsystemen toe om met het internet te communiceren via \gls{nat} maar zitten elk in een apart privénetwerk en kunnen dus niet met elkaar praten.
\textit{Host-only mode} laat gastsystemen met elkaar (en het gastheersysteem) communiceren door ze in een gezamelijk privénetwerk te plaatsen, 
maar communicatie met het internet is niet mogelijk.

\subsubsection{Softwareconfiguratie}

Hieronder leggen we kort stap voor stap uit hoe je op een Debianinstallatie de nodige SNMP software kunt installeren en configureren.
De uitleg is zeer is zeer beknopt gehouden en dient enkel om je op weg te helpen.
In de testopstelling werden de nodes ook nog geconfigureerd als switches die het \gls{stp} draaien alsook het \gls{lldp}.
Deze extra protocollen bieden informatie aan die via SNMP opgevraagd kan worden en zijn ook gegevens die in een realistische situatie opgevraagd kunnen worden.
De configuratie als switch en van LLDP wordt hier echter niet besproken. \todo{Dit kan eventueel als bijlage uitgelegd worden.}

Zoals gezegd worden alle nodes voorzien van een minimale Debian 7 installatie.
Dit wil zeggen dat er geen extra softwarepakketten worden geselecteerd bij installatie.

In de veronderstelling dat het internet werkt beginnen we na de installatie met het updaten van het systeem en het installeren van \textit{sudo}:

\begin{lstlisting}[language=bash]
# apt-get update
# apt-get upgrade
# apt-get install sudo
\end{lstlisting}

Maak een gebruiker aan en zorg ervoor dat je sudo rechten hebt met behulp van het \textit{visudo} commando of door je gebruiker toe te voegen aan de \textit{sudo} groep.

\begin{lstlisting}[language=bash]
# visudo
# usermod -a -G sudo <jouw gebruikersnaam>
\end{lstlisting}

Dan installeren we de snmp \textit{daemon} die zal antwoorden op SNMP requests.
Om te testen is het ook interessant om de client-side SNMP tools te installeren alsook een tool om de belangrijkste \glspl{mib} te downloaden.
Met die \glspl{mib} worden numerieke \glspl{oid} omgezet naar de leesbare tekstuele voorstelling.

Debian laat standaard niet toe dat niet-vrije (non-free) software\footnote{
	Dit is software die niet volledig vrij is maar op een of andere manier beperkt wordt door zijn licentie. De eisen die gesteld worden aan vrije software voor Debian zijn te vinden in de Debian Free Software Guidelines (DFSG)\cite{dfsg}\cite{dfsg-wiki}.}
geïnstalleerd wordt. De \textit{snmp-mibs-downloader} tool is daar een voorbeeld van.
Om dit toch toe te laten moet je voor elke lijn in /etc/apt/sources.list het non-free component achteraan toevoegen.
Dan krijg je zoiets:

\begin{lstlisting}[language=bash]
deb http://ftp.belnet.be/debian wheezy main non-free
deb-src http://ftp.belnet.be/debian wheezy main non-free
\end{lstlisting}

Nu kunnen we wel snmp-mibs-downloader en de andere tools installeren.

\begin{lstlisting}[language=bash]
$ sudo apt-get install snmpd snmp snmp-mibs-downloader
\end{lstlisting}

Het volgende commando zal de \glspl{mib} downloaden:

\begin{lstlisting}[language=bash]
$ sudo download-mibs
\end{lstlisting}

Het gebruik van de \glspl{mib} kun je aanzetten door de volgende regel in commentaar te zetten in het bestand /etc/snmp/snmp.conf:

\begin{lstlisting}[language=bash]
mibs :
\end{lstlisting}

Om \textit{snmpd} te configureren kun je gebruik maken van het \textit{snmpconf} commando die op interactieve wijze configuratiebestanden aanmaakt.
We moeten ook nog de locatie van de \gls{mib}-bestanden opgeven. Voeg daarom het volgende toe aan /etc/default/snmpd:

\begin{lstlisting}[language=bash]
export MIBS=/usr/share/mibs
\end{lstlisting}

Vervolgens herstarten we snmpd om de nieuwe configuratie in te laden.

\begin{lstlisting}[language=bash]
$ sudo /etc/init.d/snmpd restart
\end{lstlisting}

We kunnen testen of alles werkt met het volgende:

\begin{lstlisting}[language=bash]
$ snmpwalk -v1 -cpublic localhost mib-2
\end{lstlisting}

\todo[inline]{Verder eventueel: bridge configuratie, LLDP}

\todo[inline]{SNMP Data Retriever in een Windows XP VM?}

\todo[inline]{\LARGE {Effectieve testen mbv VirtualBox VM's}}

\todo[inline, caption={}]{Tests in VirtualBox:

\begin{itemize}
	\item Aantal SNMP objecten onder de voornaamste OID's versus het totaal aantal objecten (snmp-objecten.pl) \\
		Vergelijking tussen de verschillende iMinds switchen wat betreft aantal objecten (vergelijkfracties.pl) \\
		Zie pg.7, verslag week 9-10
	\item Reactietijd: tijd per object, SNMP walk vs bulk (verschillende groottes)
	\item SNMP Data Retriever vs. Net-SNMP
\end{itemize}

}

\subsubsection{Test \#1}


\subsection{iMinds Switches}

\todo[inline]{Betere titel! Fysieke "testopstelling" van een handvol switches. Niet echt een testopstelling natuurlijk, maar wel echte productieswitches.}


\subsection{Netwerkvertraging}

\todo[inline]{Latency speelt een grote rol, zie de onnauwkeurige resultaten. Conclusie: liefst/indien mogelijk de bevraging on-site doen, ipv remote. }

\subsection{Aantal interessante gegevens t.o.v. totaal aantal gegevens}

\todo[inline]{Is het nuttig om de op te vragen gegevens te filteren?}

\subsection{Reactietijd (VBox als iMinds switchen)}

\todo[inline]{Betere titel}

\subsection{Reactietijd voor bulkrequests}

\todo[inline]{Betere titel}

\subsection{Profiling van de SNMP Data Retriever}

\todo[inline]{Cleanup. Aanvullen na afwerken. \\
Is het een probleem dat functies en methoden door elkaar gebruikt worden?}
In deze paragraaf gaan we de SNMP Data Retriever van NetworkMining onder de loep nemen met de ingebouwde profiler van Visual Studio.
Een profiler zal ons enkele belangrijke inzichten verschaffen over wat er achter de schermen gebeurt bij het uitvoeren van de retriever.
Specifiek:

\begin{itemize}
	\item Hoe lang bepaalde stukken code er over doen
	\item Hoe vaak bepaalde stukken code uitgevoerd worden (zogenaamde hot code/paths)
	\item Problematische stukken code detecteren die er veel langer over doet dan we verwachten
	\item En bijgevolg welke stukken het meeste potentieel bieden om te optimaliseren
\end{itemize}

\subsubsection{Gebruik van de profiler in Visual Studio 2013}

De profiler in Visual Studio 2013 is erg makkelijk te gebruiken.
Om te beginnen open je je project en klik je op \emph{Analyze} in de werkbalk en kies je voor \emph{Performance and Diagnostics}.
Als alternatief kun je ook van de ALT+F2 sneltoets gebruik maken. %TODO: Koppelteken
Als \emph{Target} staat standaard het huidige project geselecteerd.
Onder \emph{Available Tools} zou normaal ook de \emph{Performance Wizard} moeten geselecteerd zijn.
Zoals de beschrijving al verklapt houdt dit onder andere het meten van CPU- en RAM-gebruik in.
\Cref{performance-wizard} toont de \emph{Performance Wizard} die je te zien krijgt als je op \emph{Start} klikt.
Hier moet je kiezen van welke profilingmethode je wenst gebruik te maken.
Wij kiezen voor de eerste: \emph{CPU Sampling}.
De rest van de stappen staan standaard goed dus mag je meteen op \emph{Finish} klikken.

\begin{figure}[]
	\centering
	\includegraphics[scale=0.50]{figures/profiler/performance-wizard}
	\caption{De Performance Wizard}
	\label{performance-wizard}
\end{figure}

Wanner het programma klaar is met uitvoeren worden de resultaten geanalyseerd.
Als dat klaar is krijg je een algemeen overzicht van de resultaten.
Eerst en vooral zie je een grafiek met het CPU-gebruik doorheen de uitvoeringstijd van de applicatie.
Daaronder zie je de \emph{Hot Paths}, dat zijn de functies die verantwoordelijk zijn voor het grootste deel van de uitvoeringstijd.
Waar wij in geïnteresseerd zijn is hieraan gerelateerd: de \emph{Call Tree View}.
Die geeft je een boomstructuur van functies die elkaar oproepen en enkele belangrijke statistieken:
hoeveel keer een functie werd opgeroepen en hoeveel tijd de functie gemiddeld nodig had om uit te voeren,
dit zowel procentueel (ten opzichte van de totale uitvoeringstijd) als in absolute tijd.

De call tree van de SNMP Data Retriever kun je zien in \cref{call-tree}.
Hierbij werd de \emph{Main} functie opengeklapt. %TODO: Koppelteken
Je kunt functies verder open klappen om te zien welke andere functies worden opgeroepen en hun aandeel in de uitvoeringstijd analyseren.
Dit kan verder gaan tot je atomaire functies krijgt die geen andere functies meer oproepen.

In de call tree zie je naast de functienaam een aantal verschillende kolommen. Hieronder volgt de lijst van de kolommen en hun betekenis.

\begin{itemize}
	\item \textbf{Number of calls:}
		dit spreekt vrij voor zichzelf. Dit is het aantal keren dat een functie opgeroepen werd.
	\item \textbf{Elapsed Inclusive Time \%:}
		dit is het percentage van de uitvoeringstijd dat werd gespendeerd in deze functie en zijn kinderen.
	\item \textbf{Elapsed Exclusive Time \%:}
		dit is het percentage van de uitvoeringstijd dat uitsluitend in deze functie werd gespendeerd, dus \emph{exclusief} zijn kinderen.
	\item \textbf{Avg Elapsed Inclusive Time:}
		dit is de gemiddelde uitvoeringstijd in milliseconden van deze functie en zijn kinderen.
	\item \textbf{Avg Elapsed Exclusive Time:}
		dit is de gemiddelde uitvoeringstijd in milliseconden van uitsluitend deze functie, dus weer \emph{exclusief} zijn kinderen.
		\todo{Check of er geen pagebreak is op de individuele items.}
\end{itemize}

\begin{figure}[h]
	\centering
	\includegraphics[scale=0.50]{figures/profiler/call-tree}
	\caption[De Call Tree]{De Call Tree. De kolommen van links naar rechts:
		\emph{Function Name},
		\emph{Elapsed Inclusive Time \%},
		\emph{Elapsed Exclusive Time \%},
		\emph{Avg Elapsed Inclusive Time},
		\emph{Avg Elapsed Exclusive Time}.}
	\label{call-tree}
\end{figure}

\subsubsection{De resultaten}

Voor het wegschrijven van de resultaten van de SNMP Data Retriever installeren we een lokale databank.
\todo{Vertellen we waarom?}
De retriever verwacht een MySQL databank, maar wij kiezen voor een MariaDB-installatie.
Omdat MariaDB een drop-in vervanging is voor MySQL is dat echter geen probleem.
Op het moment van installatie was de laatste stabiele versie 5.5.33a.

We doen een walk van 2 \glspl{oid} op een enkele productieswitch. Dit zal ons 443 objecten opleveren,
waarvan 441 die voor ons relevant zijn. Volgens de profiler doet de retriever daar ongeveer 7,4 seconden over.
De call tree van de retriever hebben we voor de leesbaarheid in een tabel gegoten (zie \cref{call-tree-main}).
De functies met een extreem kleine uitvoeringstijd (minder dan 0,01\%) hebben we achterwege gelaten.
De kolommen die je ziet zijn de functie, het aantal oproepen, de \emph{inclusieve} tijd als percentage van de totale uitvoeringstijd en
de gemiddelde \emph{inclusieve} tijd in milliseconden.

% Loglevel 2
\begin{table}[h]
	\centering
	\begin{tabular}{@{}lrrr@{}}
		\toprule
		Functie                                                  & Calls & Tijd (\%) & Tijd (ms) \\ \midrule
		SNMPDataRetrieval.RetrieveFromDevice                     & 1     & 64,39     & 4.772,57  \\
		SNMPDataRetrieval.Initialize                             & 1     & 25,37     & 1.880,45  \\
		SNMPDataRetrieval.ReadDeviceTypes                        & 1     & 4,73      & 350,46    \\
		SNMPDataRetrieval.CreateDBTable                          & 1     & 2,60      & 192,76    \\
		SNMPDataRetrieval.ReadDevices                            & 1     & 0,78      & 57,86     \\
		Microsoft.VisualBasic.CompilerServices.NewLateBinding.L… & 1     & 0,28      & 21,11     \\
		ComFunSQLConnection.GetSQLSelectData                     & 1     & 0,24      & 17,95     \\
		ComFunSQLConnection.DoSQLNonQuery                        & 1     & 0,19      & 14,18     \\
		ComFunLogger.Close                                       & 1     & 0,12      & 8,54      \\
		ComFunLogger.Log                                         & 3     & 0,08      & 1,89      \\
		ComFunSQLConnection.TableExists                          & 3     & 0,03      & 0,84      \\ \bottomrule
	\end{tabular}
	\caption{De call tree van de Main methode} %TODO: Koppelteken
	\label{call-tree-main}
\end{table}

\subsubsection{RetrieveFromDevice}

\todo[inline]{Titel: functienaam (RetrieveFromDevice) of oplossing/onderdeel/probleem (Bulk requests). \\
Hier maakt het weinig uit maar bij het volgende deel wel: Initialize of Logging (wat het probleem meteen weggeeft).}

We beginnen met de functie die het meeste tijd in beslag neemt: de \emph{RetrieveFromDevice} functie. %TODO: Koppelteken
Zoals de naam al verklapt gaat het hier om de methode die de requests stuurt om de gevraagde gegevens op te halen van de verschillende toestellen.

\todo[inline]{Hoort de uitleg over de werking van de SNMP Data Retriever hier?}
De manier waarop dit gebeurt is als volgt:
voor elk toestel dat ondervraagd moet worden wordt er een aparte thread gestart, met een maximum van 50 threads.
Elk van die threads zal alle gegevens opvragen die opgevraagd moeten worden voor dat toesteltype. \todo{Toesteltypes al vermeld?}
Wanneer een thread klaar is met gegevens opvragen wordt de thread verwijderd.
Indien er nog toestellen zijn die nog moeten ondervraagd worden, zal er dan een nieuwe thread opgestart worden voor het volgende toestel. \todo{Volledige uitleg over thread management geven of later uitleggen bij verbeteringen?}
Dit gaat zo door tot alle toestellen zijn ondervraagd.

In \cref{snmp-operaties} werden de verschillende SNMP operaties besproken waarmee men
gegevens kan opvragen. De originele versie van de SNMP Data Retriever die we eerst testen maakt gebruik van
GET-requests voor enkelvoudige gegevens en GETNEXT-requests om een SNMP walk te doen van 
een ganse deelboom.

De call tree van de RetrieveFromDevice functie zie je in \cref{call-tree-retrievefromdevice}. %TODO: Koppelteken
De twee belangrijkste methoden hier zijn de \emph{SyncRequest} en de \emph{InsertResultRow} methoden. %TODO: Koppelteken
SyncRequest maakt deel uit van de \emph{SnmpSource} bibliotheek. %TODO: Koppelteken
Dit is de third party bibliotheek waarvan gebruik gemaakt wordt om alle SNMP interacties af te handelen. %TODO: Koppelteken
De SyncRequest methode wordt gebruikt om synchroon een request te versturen.
Het feit dat de request synchroon gebeurt wil zeggen dat de code wacht op het antwoord alvorens verder te gaan.
We zien dat de methode 443 keer is opgeroepen dus dat wil zeggen dat er 443 requests verstuurd zijn geweest.
Gemiddeld deed een request er een kleine 10 milliseconden over, allen samen goed voor bijna 60\% (oftewel 4,28 seconden) van de totale uitvoeringstijd.

\begin{table}[h]
	\centering
	\begin{tabular}{@{}lrrr@{}}
		\toprule
		Functie                                                  & Calls & Tijd (\%) & Tijd (ms) \\ \midrule
		SnmpSource.SnmpSession.SyncRequest                       & 443   & 57,74     & 9,66      \\
		SNMPDataRetrieval.InsertResultRow                        & 441   & 3,75      & 0,63      \\
		SnmpSource.SnmpSession..ctor                             & 1     & 1,85      & 137,38    \\
		Microsoft.VisualBasic.CompilerServices.NewLateBinding.L… & 445   & 0,42      & 0,07      \\
		ComFunSQLConnection.TableContainsColumn                  & 2     & 0,12      & 4,43      \\
		ComFunLogger.Log                                         & 451   & 0,07      & 0,01      \\
		SnmpSource.SnmpPdu..ctor                                 & 1     & 0,05      & 3,84      \\
		Microsoft.VisualBasic.CompilerServices.Conversions.ToBo… & 2     & 0,04      & 1,59      \\
		Microsoft.VisualBasic.CompilerServices.NewLateBinding.L… & 6     & 0,04      & 0,52      \\
		SnmpSource.SnmpVariable.CreateSnmpVariable               & 443   & 0,02      & 0,00      \\ \bottomrule
	\end{tabular}
	\caption{De call tree van de RetrieveFromDevice methode} %TODO: Koppelteken
	\label{call-tree-retrievefromdevice}
\end{table}

\todo[inline]{Theoretische resultaten? (Adhv. testen met VirtualBox \& iMinds Switchen)}

\todo[inline]{Aanvullen. Implementeren van BULK requests.}

\subsubsection{Initialize}
\todo{Titel ok?}
\todo[inline]{Herschrijf begin van stuk over logging framework.}
Het eerste wat er ons opvalt als we de call tree bekijken is dat de \emph{Initialize} methode 25\% van de
uitvoeringstijd voor zijn rekening neemt.
Rijst de vraag wat deze functie juist doet dat ze zoveel tijd nodig heeft.

We klappen de Initialize methode open om de boosdoeners te zoeken.
De functies van de Initialize methode vind je terug in \cref{call-tree-initialize}.
We zien twee oproepen naar een \emph{Log} methode die er gemiddeld bijna een seconde over doet \emph{per oproep}! %TODO: Koppelteken

\begin{table}[h]
	\centering
	\begin{tabular}{@{}lrrr@{}}
		\toprule
		Functie                                                   & Calls & Tijd (\%) & Tijd (ms) \\ \midrule
		ComFunLogger.Log                                          & 2     & 20,97     & 777,30    \\
		ComFunSQLConnection..ctor                                 & 1     & 3,51      & 259,97    \\
		ComFunLogger.set\_LogFile                                 & 1     & 0,19      & 13,87     \\
		System.Configuration.ConfigurationManager.get\_AppSettin… & 12    & 0,16      & 0,99      \\
		Microsoft.VisualBasic.CompilerServices.Conversions.ToIn…  & 2     & 0,15      & 5,58      \\
		ComFunLogger.Log                                          & 4     & 0,13      & 2,41      \\
		ComFunLogger.Log                                          & 3     & 0,06      & 1,59      \\
		ComFunLogger.Log                                          & 2     & 0,05      & 1,87      \\
		ComFun.NetworkMiningCopyRightStatement                    & 1     & 0,03      & 2,03      \\
		ComFunLogger..ctor                                        & 1     & 0,01      & 1,10      \\ \bottomrule
	\end{tabular}
	\caption{De call tree van de Initialize methode} %TODO: Koppelteken
	\label{call-tree-initialize}
\end{table}

De \emph{ComFunLogger} is een stuk code die gebruikt wordt om te loggen naar een tekstbestand.
De naam komt van het feit dat ze een gemeenschappelijk stuk code is die over meerdere projecten kan
gebruikt worden: \emph{common functions}, of \emph{ComFun} voor kort.
Maar een functie die loggegevens wegschrijft naar een bestand hoort niet zo lang te duren.
I/O operaties zijn kostelijk, maar niet \emph{zo} kostelijk. %TODO: Koppelteken
Als we de functie helemaal openklappen in \cref{call-tree-performancecounter} vinden we de echte dader:
een constructor van \emph{System.Diagnostics.PerformanceCounter}.
Een performance counter wordt gebruikt voor het monitoren van systeemcomponenten zoals
processoren, geheugen en netwerk I/O. Als je ze gebruikt in je applicatie kunnen ze je %TODO: Koppelteken
informatie geven over de performantie van je programma.\cite{performance-counters-intro}
De ComFunLogger gebruikt ze om het geheugengebruik te meten en weg te schrijven in de logbestanden.

\begin{figure}[h]
	\centering
	\includegraphics[scale=0.50]{figures/profiler/call-tree-performancecounter}
	\caption{De call tree van de Log methode}
	\label{call-tree-performancecounter}
\end{figure}

\todo[inline]{Herlees mij.}

Als oplossing werd ervoor gekozen om een alternatieve \emph{logging framework} te gebruiken.
Dit lijkt een drastische maatregel (en dat is het ook), maar daar zijn goede redenen voor.
Er zijn een heleboel gratis en open-source logging frameworks die reeds hun nut en kunnen bewezen hebben.
Zij zijn \emph{tried and true} oplossingen voor een algemeen probleem. \todo{Het zijn...}
Het is dus ook in de beste interesse voor een bedrijf om hiervoor te kiezen. \todo{Correcte zin?}
Financiëel is het een goede oplossing want de software is al ontwikkeld en is bewezen dat ze werkt.
Dit spaart tijd en geld uit voor de ontwikkeling van een eigen oplossing.
De software is gratis in gebruik dus er zijn geen licentiekosten aan verbonden.
Er zijn ook geen of lage onderhoudskosten. De software wordt al ingezet in zeer diverse omgevingen dus is al zeer uitgebreid.
De kans dat de software een bepaalde functionaliteit mist is dus klein.
En als er iets moet toegevoegd worden beschik je ook over de broncode.

Ook vanuit technisch opzicht is het een goede keuze.
Zoals gezegd heeft de software al zijn nut bewezen in diverse omgevingen.
In het specifieke geval van de ComFunLogger biedt een \emph{third party} oplossing ook een heleboel
extra flexibiliteit en features. Zo kan je loggen naar meerdere outputs en zijn er meerdere mogelijke outputs beschikbaar.
Je kunt bijvoorbeeld loggen naar tekstbestanden, consolevensters, databanken, enzovoort.

De twee belangrijkste redenen echter zijn het feit dat ze ontwikkeld zijn om een zo klein mogelijke performantieimpact te hebben en
dat de implementatie zeer simpel is.
Zo was het veel sneller om een ander logging framework te gebruiken dan om bekend te raken met de bestaande loggingcode en die te optimaliseren.

De keuze is uiteindelijk gevallen op Apache log4net en is gebaseerd op waarschijnlijk het bekendste logging framework voor java: Apache log4j.
Bij de keuze werd rekening gehouden met de performantieimpact en de features van de verschillende logging frameworks.\footnote{
	Een vergelijking tussen de bekendste logging frameworks voor .NET vind je 
	terug in de bronnenlijst bij bron \cite{logging-frameworks-and-performance} en \cite{logging-frameworks}.}



\todo[inline, caption={}]{

\begin{itemize}
	\item Waarom zijn er twee instanties nodig van de PerformanceCounter?
	\item Onderzoek op het internet leert ons dat PerformanceCounters hele kostelijke objecten zijn om aan te maken. Bron/citaat?
	\item x Oplossing: alternatieve loggingframework. Maar waarom heb je hiervoor gekozen ipv de huidige aan te passen?
	\item x Performantieredenen: ik heb een tried \& true logging framework opgezocht met nadruk op een minimale performantieimpact.
	\item x Plus de implementatie is ook sneller. Dan moet ik niet mijzelf bekend maken met de oude loggingcode en heb ik maar het 
		nieuwe loggingframework te includeren en alle logcalls te vervangen, wat vrij snel gebeurd is.
	\item x Lagere onderhoudskosten
	\item x Geen licentiekosten
	\item x Als leuke bonus krijg je er ook een heleboel extra features bij zoals logging naar meerdere outputs en meer outputformaten. (extra flexibiliteit)
	\item x Denk aan textfile, XML file, DB file, consoleuitvoer, etc.
	\item Nadeel: hoe groot is de extra code/binary van dit loggingframework? Andere nadelen?
\end{itemize}

}

\subsection{Tabellen rij per rij opvragen i.p.v. kolom per kolom}

\todo[inline]{Betere/kortere titel?}

\todo[inline, caption={}]{Vraag: waarom zou dit sneller zijn??? \\
Als je graag rij per rij LEEST zou kan je gebruik maken van het \textbf{smptable commando}.
Deze zorgt voor de correcte alignering en houdt rekening met indexen en gaten in de tabel.
Zie ook \textbf{http://www.net-snmp.org/wiki/index.php/TUT:snmptable}. \\
Zowel op VBox als iMinds Switchen \\

Caveats:

\begin{itemize}
	\item Goede integratie van MIB's vereist, want je moet weten welke kolommen er zijn en wat de indexering is
	\item Rekening houden met lege cellen/gaten in de tabel
	\item Als je de indexen op voorhand niet kent kun je maar een rij tegelijkertijd opvragen m.b.v. getnexts,
		Je kunt dus niet meer gegevens opvragen als er kolommen zijn in een request. Bulkrequests hebben daarentegen
		geen limiet op het aantal objecten (afgezien van de pakketgrootte natuurlijk).
\end{itemize}

}

\subsection{Impact van de fragmentatie van pakketten}

\todo{Titel: Snelheidsimpact?}


\section{Grootschalige benchmarks en experimenten}

\subsection{Testopstelling}

\todo[inline]{Testopstelling op de Virtual Wall.}

\subsection{Impact databankinteracties}

\subsection{Impact fragmentatie}

\subsection{Benchmarks uitvoeringstijd}

\todo[inline]{Oude versie vs. nieuwe versie}

\subsection{Benchmarks bandbreedte}

\subsubsection{SNMP Walk versus SNMP Bulk Requests}

\subsubsection{Invloed aantal ondervraagde nodes}

\todo[inline]{Verwijs voor dip in bandbreedte bij groot aantal nodes (>50) naar sectie CPU gebruik.}

\subsection{Benchmarks CPU-gebruik}

\todo[inline]{Onderzoek waarom verkeer \& CPU-verbruik dippen voor het einde (aantal threads daalt).}

\subsection{Benchmarks geheugenverbruik}

% Optimalisaties (Codeoptimalisaties / Optimalisaties voor de SNMP Data Retriever / SNMP Data Retriever Optimalisaties)

% Problemen
\chapter{Problemen}

\todo[inline]{Zou dit niet beter een apart hoofdstuk zijn?}

\todo[inline,caption={}]{

\begin{itemize}

	\item Kolombreedte OID's (en andere? Hostnaam?) te klein
	\item endOfMibView exceptie niet ondersteund door SNMP Data Retriever \\
		Zie pg. 7, verslag week 9-10
	\item Grootschalige testopstelling op de Virtual Wall \\
		Probleem met de virtualisatie (beta achtige feature, werkt in de praktijk niet zo goed)
		Problemen met netwerkopstelling/adapters voor virtuele nodes
	
\end{itemize}

Niet opgelost/niet belangrijk:

\begin{itemize}

	\item DDOS-bescherming bij iMinds
	\item Permissieproblemen met snmpd.	$ \rightarrow $ IP range beperking~
	\item Omdat mijn reactietijden.pl script maar een commando uitvoert, hebben we voor het benchmarken van bulkrequests die manueel opgesteld worden een ander script nodig.
	Ik heb me hiervoor gebaseerd op m'n benchmarking script voor Windows die zelf de tijden meet (omdat ik de /usr/bin/time van Linux niet heb in Windows).
	Deze maakt gebruik van een CPAN module. Om zeker te zijn dat er geen grote verschillen tussen de timing-methoden zitten, heb ik nog een derde manier geprobeerd.
	Ik maak gebruik van HiRes time en hou start- en eindtijd bij en geef het verschil terug. De drie timingmethoden komen dicht genoeg bij elkaar in de buurt om te concluderen
	dat er geen timingverschillen optreden bij het gebruik van verschillende timingmethoden en ik dus de tijden kan vergelijken. \\
	Zie ook meerderekolommen - Vergelijk timings.pl en Verschillen in benchmark timings.txt.
	\item Totaal aantal objecten dat een agent aanbiedt.
	\item Gaten in tabellen, zie random notes.txt
	\item Problemen met verouderde softwarepakketten op Debian
	
\end{itemize}

}


\section{Ondersteuning voor lange OID's}

\section{Ondersteuning voor de endOfMibView exceptie}

\section{''DOS-bescherming'' op intern netwerk iMinds}

\todo[inline]{Betere titel? Alhoewel de oorzaak nooit is vastgelegd...}

\section{Grootschalige opstelling op de Virtual Wall m.b.v. virtualisatie}

\todo[inline]{Betere/kortere titel?}

% Voorstellen voor verdere verbeteringen
\chapter{Voorstellen voor verdere optimalisaties}

\todo[inline]{Inleiding?}

\section{Bulk inserts bij databankinteracties}

De databankinteracties werden uitvoerig besproken in de \cref{werking,profiling,impact-db}.
Een verdere optimalisatie die zeker nog gedaan moet worden,
is het gebruik van bulk inserts bij het wegschrijven van de opgehaalde gegevens in de databank.

In \cref{impact-db} hebben we gezien dat het wegschrijven van de gegevens in de databank een gigantische impact heeft op de uitvoeringstijd,
dat alleen maar toeneemt met het aantal te bevragen toestellen.
Het bevragen van slechts 100 toestellen ging zonder het wegschrijven van de resultaten maar liefst 61 keer sneller.
Sterker nog, de impact van de databankinteracties is zodanig groot dat de performantiewinst van de andere optimalisaties bijna teniet gedaan wordt.
De nieuwe versie was bij 100 toestellen slechts 5\% sneller dan de oude, maar ruim zeven keer sneller als de gegevens niet weggeschreven moesten worden.
Het implementeren van bulk inserts is dus absoluut cruciaal om de \nwmretriever{} succesvol te kunnen inzetten op grote schaal.

Het implementeren van bulk requests hoeft zelfs niet zoveel werk en tijd te vereisen.
Een eenvoudige implementatie houdt een globale tabel bij in het geheugen die de resultaten tijdelijk bijhoudt.
De InsertResultRow-methode wordt dan aangepast om de gegevens in die tabel in het geheugen weg te schrijven in plaats van rechtstreeks in de databank.
Eenmaal een bepaalde hoeveelheid gegevens in de databank weggeschreven zijn kan de InsertResultRow-methode dan beslissen om al die data
in een keer weg te schrijven in de databank met dus een bulk insert.

Het nadeel van deze manier van werken is dat het wegschrijven van de resultaten in dezelfde thread gebeurt als het bevragen van een toestel.
Als de gegevens weggeschreven worden zal die thread en het bevragen van dat ene toestel een stuk langer duren dan anders.
Een andere manier van werken die wat meer arbeidsintensief is en meer tijd vraagt werd ook al voorgesteld in \cref{impact-db}.

Daarbij wordt de verantwoordelijkheid van het wegschrijven van de resultaten in de databank overgeheveld naar een aparte thread.
Deze zal dan instaan om, wanneer de tabel in het geheugen een bepaalde hoeveelheid gegevens bevat de gegevens weg te schrijven naar de databank,
zonder de andere retrieverthreads daarbij te storen.
De InsertResultRow-methode is dan enkel nog verantwoordelijk voor het wegschrijven van de gegevens in de tabel in het geheugen.

Men kan dan nagaan of het beter is om periodiek gegevens weg te schrijven tijdens het bevragen van de toestellen,
of om de gegevens ineens weg te schrijven na het bevragen van de toestellen, of naar het einde toe als het aantal retrieverthreads daalt.

De gegevens hoeven zelfs niet weggeschreven te worden in een databank.
Men kan verschillende implementaties voorzien die de tabel in het geheugen op verschillende manieren verwerkt.
Men kan er bijvoorbeeld ook voor kiezen om de gegevens naar een bestand weg te schrijven of een combinatie.
Het grote voordeel is dat het opslaan van de gegevens los staat van het opvragen ervan, dus dat de snelheidsimpact ervan beperkt blijft.


\section{Intelligent threadbeheer}

Het beheer van threads en het CPU-gebruik werden besproken in \cref{werking,cpu-gebruik}.

Er was al gebleken dat het beheer van de threads in de originele versie van de \nwmretriever{} beter kon.
Het probleem was dat er gewacht werd tot de eerst gestarte thread klaar was alvorens nieuwe threads aan te maken voor het bevragen van toestellen.
Dat leidde tot korte periodes waarbij het aantal retrieverthreads zakte en er geen neiuwe threads werden aangemaakt.
Maar zoals gezegd werd dit probleem reeds aangepakt door NetwerkMining tijdens de masterproef en
wordt er gebruik gemaakt van een andere implementatie in de nieuwe versies van de \nwmretriever{}.

We hebben ook gezien dat het aantal threads vaststond op 50 threads.
Een betere manier van werken zou het aantal threads dynamisch bepalen aan de hand van de beschikbare resources (zowel CPU-, RAM- als bandbreedtegebruik).
Uit de tests in \cref{benchmarks-geheugengebruik,benchmarks-bandbreedte} bleek echter dat het geheugen- en bandbreedtegebruik beperkt blijft
en er dus meer dan genoeg reserve over is voor die resources.
Het aantal threads zou dus voornamelijk afhangen van het CPU-gebruik.

Op de server van de Virtual Wall was er nog een overschot aan CPU-rekenkracht,
maar het omgekeerde is ook mogelijk.
Bij een tekort aan rekenkracht heeft het geen zin om meer threads aan te maken en kan het mogelijk zelfs een averechts effect hebben.

Het dynamisch beheren van threads is een algemeen probleem waar veel softwareprojecten mee geconfronteerd worden.
Gelukkig betekent dit ook dat er reeds vele oplossingen voor zijn ontwikkeld die vrij te gebruiken zijn.
Het .NET-platform biedt hiervoor een eigen oplossing onder de vorm van de \gls{tpl} sinds versie 4 van het .NET-framework.

Het doel van \gls{tpl} is om het ontwikkelaars gemakkelijker te maken om gebruik te maken van parallelisme en multithreading in applicaties.
\Gls{tpl} schaalt dynamisch de mate van parallelisme om zo efficiënt mogelijk gebruik te maken van alle processorcores die beschikbaar zijn.
\Gls{tpl} verzorgt ook het inplannen van threads in een \textit{thread pool}, ondersteuning voor het annuleren van threads, toestandbeheer en
andere low-level details\cite{msdn-tpl}.


% Testopstelling
%TODO: DELETE ME
\chapter{Testopstelling}


\section{Gebruikte software}

\subsection{NuDesign Visual MIBrowser}
De basic versie is een MIB browser die gratis te gebruiken is.

\subsection{Net-SNMP}

\subsubsection{Client-side tools}
De commandline tools voor SNMP requests.

\subsubsection{Server-side tools}
snmpd: de Net-SNMP agent.

\subsection{lldpd}
\todo[inline]{Schrappen.}
lldpd is een implementatie van LLDP voor Unix met ook ondersteuning voor verschillende proprietary protocols.


\section{Eerste testopstelling}
Testopstelling van Wouter Tavernier

\section{Netwerkopstelling VirtualBox}
Kleinschalig netwerkopstelling met een management netwerk en een meshnetwerk dat de switches onderling verbindt.
Maakt gebruik van een nieuwe netwerkfeature in VirtualBox die, behalve in een nieuwsupdate bij de release ervan, nog niet gedocumenteerd is.

\section{Netwerkopstelling Virtual Wall}
Een grotere opstelling op de Virtual Wall.


\section{Configuratie Debian machines}

\subsection{Net-SNMP}
Configuratie van Net-SNMP en vooral de snmpd.

\subsubsection{MIB's}
Installatie van de MIB's.

\subsection{IP-configuratie}
Configuratie voor de verschillende netwerk(interfaces).

\subsection{VirtualBox klonen}
\todo[inline]{Schrappen.}
Opzetten van gekloonde VM's die gelinkt zijn aan het origineel.

\subsubsection{VirtualBox crashes}
\todo[inline]{Schrappen.}
Bij het minste dat er verkeerd gebeurt bij het afsluiten zijn al je VM's omzeep...

\subsection{Bridge configuratie}
Configuratie van de bridge/switch.

\subsubsection{BRIDGE-MIB}
Agent implementatie voor de BRIDGE-MIB.

\subsection{lldpd}
\todo[inline]{Schrappen.}
Configuratie van lldpd.

\subsubsection{Problemen met Net-SNMP agent software}
Tal van problemen bij de configuratie van snmpd voor het ondersteunen van de LLDP-MIB agent implementatie.

\subsubsection{MIB's}
LLDP-MIB is moeilijker te vinden: verschillende versies, verschillende vendors, ...
Welke is de juiste?

\subsection{Simulatie van netwerkvertraging}
Netwerkvertraging simuleren.

% Meetresultaten/oplossingen
%TODO: DELETE ME
\chapter{Meetresultaten en oplossingen}

\section{Testmethode}

\subsection{CPU}
Hoe CPU metingen uitvoeren? Beperking: slechts een meting per seconde.

\subsection{Uitvoeringstijd}
Hoe uitvoeringstijd meten? Beperking: nauwkeurigheid.

\subsection{RAM?}
Hoe RAM meten? Is dit nodig? Tot zover nog geen problematische RAM-vereisten opgemerkt.

\subsection{Bruikbare OID's}
\todo[]{Betere titel?}
\todo[caption=Bruikbare-OIDs]{Dit kan hier uitgelegd worden als het invloed heeft op de testresultaten. Zoniet kan dit evt elders uitgelegd worden.}
Niet alle OID's die opgevraagd zijn, zijn nuttig. Bij een SNMP walk heb je altijd een OID meer dan nodig.
Ook bij GETBULK requests heb je overtollige OID's.


\section{Beginsituatie}

\subsection{Problemen kolombreedte}
Zie problemen met te lange OID's.

\section{Specifieke oplossing}
Een voorbeeld van een voorstel voor een oplossing met de puntjes die daarbij aan bod moeten komen.

\subsection{Beschrijving}
Beschrijving idee, redenering/denkpiste, hypothese.

\subsection{Implementatie}
Code, uitleg.

\subsection{Resultaten}
Meetresultaten.

\subsection{Conclusies}
Goede oplossing of niet?


\section[GETBULK-operatie]{Gebruik van GETBULK-operatie}
Een eerste voorstel voor een oplossing is alvast het gebruik van de GETBULK-operatie.


\section{Vervanging loggingframework}
\label{vervanging-loggingframework}

\todo[inline,caption=Vervanging-loggingframework]{De retriever maakte gebruik van een logging framework dat in-house was ontwikkeld. Bij het profilen van de applicatie
bleek echter dat er performantieproblemen waren bij dat framework, specifiek door de "count objecten". Het framework is dan vervangen
geweest door een bestaande, open-source alternatief na analyse van de performantie en features. (log4net - gebaseerd op log4j, beiden
te vinden op de Apache Foundation.)}


\section{Beter beheer van threads}
Wanner max \# taken/threads bereikt, wordt er gewacht op de eerste thread om af te ronden.
Als de eerste thread vastzit, zit je met een probleem. Mss beter om alle threads te overlopen en te kijken welke vrij is. \\
Hergebruik van threads mbv threadpool?

De winst hiervan kan groot zijn, maar enkel in uitzonderlijke omstandigheden. Onder normale omstandigheden zal dit wss geen probleem vormen.

\todo[inline]{Dit is in de loop van de thesis geïmplementeerd geweest door NetworkMining (Leo).}


\section{Optimalisatie databankverkeer}
\todo[inline,caption=DB-interacties]{
Worden resultaten gecachet alvorens ze worden weggeschreven in de DB? Of worden ze een voor een weggeschreven? \\
Moet nog eens  gecontroleerd worden maar de devices tabel leek bij vlug overzicht te worden ingelezen uit het XML bestand,
werd dan naar DB weggeschreven en dan weer opnieuw uit DB uitgelezen.

Dit is onderzocht en alhoewel de DB-interacties verre van efficient zijn, is de impact op de totale uitvoeringstijd verwaarloosbaar.

Alhoewel! Dit bleek een probleem te zijn. Niet opgemerkt vanwege SSD.}

\section{Verbeterde integratie MIB-bestanden}
Gegevens worden gekoppeld aan datatype mbv naam en niet OID. Vervelend voor tabellen vermits de namen voor kolommen dan niet getoond worden.
Is meer een verbetering op vlak van usability, en niet op performance. Valt buiten de scope van de masterproef maar kan als aanbeveling dienen naar NetworkMining toe.

% Tekstfragmenten
%TODO: DELETE ME
\chapter{Tekstfragmenten}

\todo[inline]{Tekstfragmenten verwijderen.}

Dit hoofdstuk bevat enkel losse stukken tekst die nog niet in de scriptie op hun uiteindelijke plaats zijn gezet.

\section{DB installatie op SSD}

> Lokale DB installatie. \\
De reden hiervoor was omdat we de reeds geïnstalleerde databank op de virtuele machine niet konden bereiken.
Door onoplettendheid zijn we echter vergeten dat de lokale schijf een SSD is.
SSD's bieden databanken een enorme performantiewinst over klassieke mechanische harde schijven.
Dit zal er toe leiden dat we aanvankelijk de verkeerde conclusie trekken dat de databankinteracties
een verwaarloosbare impact hebben op de totale uitvoeringstijd.


% Conclusie / Besluit
% (antwoord op centrale vraag, korte samenvatting, geen nieuwe elementen …)
\chapter{Conclusie}


% Referentielijst
% Include all references. Replace * with reference key to include specific references only.
%\nocite{*}
%\nocite{snmp-wiki}
\printbibliography[heading=bibintoc]

% Bijlages (facultatief)
\begin{appendices}


\chapter{Een bijlage}

\todo[inline]{Fix bijlagen.}
Inhoud van bijlage.


\chapter{Nog een bijlage}

Inhoud van tweede bijlage.


\end{appendices}

% Blanco blad
\newpage
\null
\thispagestyle{empty}
\newpage

% Onbedrukte kaft (licht karton, zelfde kleur als bedrukte kaft)

\end{document}