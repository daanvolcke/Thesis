% Woordenlijst

\newglossaryentry{arptabel}
{
	name=ARP-tabel,
	description={ARP staat voor het Address Resolution Protocol.
				Met behulp van ARP kan een toestel het unieke hardware-adres (MAC-adres) te weten komen dat bij een IP-adres hoort van een ander toestel.\cite{arp-nlwiki}}
}


% Afkortingen

\newacronym{mdb}{MDB}{Manageable Objects Database}
\newacronym{oid}{OID}{Object Identifier}
\newacronym{mib}{MIB}{Management Information Base}
\newacronym{pdu}{PDU}{Protocol Data Unit}
\newacronym{nat}{NAT}{Network Address Translation}
\newacronym{stp}{STP}{Spanning Tree Protocol}
\newacronym{lldp}{LLDP}{Link Layer Discovery Protocol}
\newacronym{mtu}{MTU}{Maximum Transmission Unit}
\newacronym{tlv}{TLV}{Tag-Length-Value}
\newacronym{lxc}{LXC}{Linux Containers}
\newacronym{cgroups}{cgroups}{control groups}

% Afkortingen met afwijkend meervoud

%\newacronym{nms}{NMS}{Network Management Systeem}
\newglossaryentry{nms}
{
	name=NMS,
	description={Network Management Systeem},
	first={Network Management Systeem (NMS)},
	plural=NMS'en,
	longplural={Network Management Systemen},
	type=\acronymtype
}