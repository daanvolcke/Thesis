% Woordenlijst

\newglossaryentry{arptabel}
{
	name=ARP-tabel,
	description={ARP staat voor het Address Resolution Protocol.
				Met behulp van ARP kan een toestel het unieke hardware-adres (MAC-adres) te weten komen dat bij een IP-adres hoort van een ander toestel.\cite{arp-nlwiki}}
}

\newglossaryentry{snmp-agent}
{
	name=SNMP-agent,
	description={Een SNMP-agent vervult de serverrol bij SNMP-protocol.
		Ze houdt de informatie bij over een toestel en beantwoordt requests die om die informatie vragen.
		Zie ook \cref{snmp-rollen}.}
}

\newglossaryentry{snmp-object}
{
	name=SNMP-object,
	description={\ldots}
}

\newglossaryentry{oid-explained}
{
	name=OID,
	description={Identificeert een boomtak die een of meerdere objecten bevat.}
}

\newglossaryentry{net-snmp-tools}
{
	name={Net-SNMP commandlinetools},
	description={\ldots}
}


% Afkortingen

\newacronym{mdb}{MDB}{Manageable Objects Database}
\newacronym{oid}{OID}{Object Identifier}
\newacronym{mib}{MIB}{Management Information Base}
\newacronym{pdu}{PDU}{Protocol Data Unit}
\newacronym{nat}{NAT}{Network Address Translation}
\newacronym{stp}{STP}{Spanning Tree Protocol}
\newacronym{lldp}{LLDP}{Link Layer Discovery Protocol}
\newacronym{mtu}{MTU}{Maximum Transmission Unit}
\newacronym{tlv}{TLV}{Tag-Length-Value}
\newacronym{lxc}{LXC}{Linux Containers}
\newacronym{cgroups}{cgroups}{control groups}
\newacronym{dos}{DoS}{Denial of Service}
\newacronym{tpl}{TPL}{Task Parallel Library}

% Afkortingen met afwijkend meervoud

%\newacronym{nms}{NMS}{Network Management Systeem}
\newglossaryentry{nms}
{
	name=NMS,
	description={Network Management Systeem},
	first={Network Management Systeem (NMS)},
	plural=NMS'en,
	longplural={Network Management Systemen},
	type=\acronymtype
}