\chapter{Meetresultaten en oplossingen}

\section{Testmethode}

\subsection{CPU}
Hoe CPU metingen uitvoeren? Beperking: slechts een meting per seconde.

\subsection{Uitvoeringstijd}
Hoe uitvoeringstijd meten? Beperking: nauwkeurigheid.

\subsection{RAM?}
Hoe RAM meten? Is dit nodig? Tot zover nog geen problematische RAM-vereisten opgemerkt.

\subsection{Bruikbare OID's}
\todo[]{Betere titel?}
\todo[caption=Bruikbare-OIDs]{Dit kan hier uitgelegd worden als het invloed heeft op de testresultaten. Zoniet kan dit evt elders uitgelegd worden.}
Niet alle OID's die opgevraagd zijn, zijn nuttig. Bij een SNMP walk heb je altijd een OID meer dan nodig.
Ook bij GETBULK requests heb je overtollige OID's.


\section{Beginsituatie}

\subsection{Problemen kolombreedte}
Zie problemen met te lange OID's.

\section{Specifieke oplossing}
Een voorbeeld van een voorstel voor een oplossing met de puntjes die daarbij aan bod moeten komen.

\subsection{Beschrijving}
Beschrijving idee, redenering/denkpiste, hypothese.

\subsection{Implementatie}
Code, uitleg.

\subsection{Resultaten}
Meetresultaten.

\subsection{Conclusies}
Goede oplossing of niet?


\section[GETBULK-operatie]{Gebruik van GETBULK-operatie}
Een eerste voorstel voor een oplossing is alvast het gebruik van de GETBULK-operatie.


\section{Vervanging loggingframework}
\label{vervanging-loggingframework}

\todo[inline,caption=Vervanging-loggingframework]{De retriever maakte gebruik van een logging framework dat in-house was ontwikkeld. Bij het profilen van de applicatie
bleek echter dat er performantieproblemen waren bij dat framework, specifiek door de "count objecten". Het framework is dan vervangen
geweest door een bestaande, open-source alternatief na analyse van de performantie en features. (log4net - gebaseerd op log4j, beiden
te vinden op de Apache Foundation.)}


\section{Beter beheer van threads}
Wanner max \# taken/threads bereikt, wordt er gewacht op de eerste thread om af te ronden.
Als de eerste thread vastzit, zit je met een probleem. Mss beter om alle threads te overlopen en te kijken welke vrij is. \\
Hergebruik van threads mbv threadpool?

De winst hiervan kan groot zijn, maar enkel in uitzonderlijke omstandigheden. Onder normale omstandigheden zal dit wss geen probleem vormen.

\todo[inline]{Dit is in de loop van de thesis geïmplementeerd geweest door NetworkMining (Leo).}


\section{Optimalisatie databankverkeer}
\todo[inline,caption=DB-interacties]{
Worden resultaten gecachet alvorens ze worden weggeschreven in de DB? Of worden ze een voor een weggeschreven? \\
Moet nog eens  gecontroleerd worden maar de devices tabel leek bij vlug overzicht te worden ingelezen uit het XML bestand,
werd dan naar DB weggeschreven en dan weer opnieuw uit DB uitgelezen.

Dit is onderzocht en alhoewel de DB-interacties verre van efficient zijn, is de impact op de totale uitvoeringstijd verwaarloosbaar.

Alhoewel! Dit bleek een probleem te zijn. Niet opgemerkt vanwege SSD.}

\section{Verbeterde integratie MIB-bestanden}
Gegevens worden gekoppeld aan datatype mbv naam en niet OID. Vervelend voor tabellen vermits de namen voor kolommen dan niet getoond worden.
Is meer een verbetering op vlak van usability, en niet op performance. Valt buiten de scope van de masterproef maar kan als aanbeveling dienen naar NetworkMining toe.