\chapter*{Conclusie}
\addcontentsline{toc}{chapter}{Conclusie}

Bij de aanvang van deze masterproef werd de \nwmretriever{} enkel ingezet op kleine- en middelgrote netwerken.
We hebben de belangrijkste aspecten van het SNMP-retrievalproces doorgelicht en hebben daarbij een aantal problemen blootgelegd die
het gebruik van de retriever op grootschalige netwerken belemmerden.
We hebben de nodige aanpasingen aangebracht aan de retriever om die problemen op te lossen,
met name het gebruik van bulkrequests en het gebruik van een alternatief loggingframework,
en konden besluiten dat de schaalbaarheid sterk verbeterd is.

Er is echter nog een cruciaal probleem die opgelost moet worden alvorens we kunnen besluiten dat de \nwmretriever{} schaalbaar genoeg is om in te kunnen zetten op grootschalige netwerken: de databankinteracties.
Er zijn verschillende manieren waarop dit probleem aangepakt kan worden,
en de eenvoudigste zijn gelukkig niet al te ingrijpend en hoeven niet al te veel tijd in beslag te nemen.

Zij we geslaagd in ons opzet om de schaalbaarheid te onderzoeken en te verbeteren?
Alhoewel er nog wat werk te doen is op het gebied van databankinteracties, kunnen we daar wel al positief op antwoorden:
als we niet tegengehouden worden door de databankinteracties, dan hebben we de \nwmretriever{} ruim zeven keer sneller gemaakt bij het bevragen van 100 toestellen.
Een theoretische oefening op de uitvoeringstijd bij nog grotere aantallen toestellen leerde ons dat de impact van extra te bevragen toestellen beperkt blijft.

Ook het geheugen- als bandbreedtegebruik blijft zeer minimaal bij het bevragen van een groot aantal toestellen.
Voor wat betreft het CPU-gebruik konden we wel nog besluiten dat er nog verbeteringen mogelijk zijn bij het beheren van het aantal threads.


\todo[inline,caption={}]{

\begin{itemize}
	\item Doel masterproef herhalen
	\item Resultaten bespreken.
	\item Door gebruik van bulk en log4net \ldots
	\item Geslaagd in opzet?
	\item 
	\item Samenvatting van de conclusies van de tests.
	\item Verdere optimalisaties.
	\item 
	\item Vraag: is de retriever uiteindelijk schaalbaar genoeg?
	\item 
	\item antwoord op centrale vraag, korte samenvatting, geen nieuwe elementen \ldots
\end{itemize}

}