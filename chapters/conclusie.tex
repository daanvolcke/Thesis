\chapter*{Conclusie}
\addcontentsline{toc}{chapter}{Conclusie}

Bij de aanvang van de masterproef werd de \nwmretriever{} enkel ingezet op kleine- en middelgrote netwerken.
We hebben de belangrijkste aspecten van het retrievalproces doorgelicht en hebben daarbij de belangrijkste problemen blootgelegd die
het gebruik van de retriever op grootschalige netwerken belemmerden.
We hebben de nodige aanpasingen aangebracht aan de retriever om die problemen op te lossen,
waaronder het gebruik van bulkrequests en de implementatie van een alternatief loggingframework.
Met die aanpassingen konden we besluiten dat de schaalbaarheid van de applicatie er sterk op vooruit is gegaan.

Er is echter nog een cruciaal probleem dat opgelost moet worden alvorens we de \nwmretriever{} kunnen aanbevelen voor het gebruik ervan op grootschalige netwerken:
de databankinteracties.
We hebben verschillende manieren besproken waarop dit probleem aangepakt kan worden,
en de eenvoudigste zijn gelukkig niet al te ingrijpend en hoeven niet al te veel tijd in beslag te nemen.

Zijn we geslaagd in onze opzet om de schaalbaarheid te onderzoeken en te verbeteren?
Zeker, alhoewel er wel nog wat werk te doen is op het gebied van databankinteracties.
Als we niet vertraagd zouden worden door de databankinteracties, dan hebben we de \nwmretriever{} ruim zeven keer sneller gemaakt bij het bevragen van 100 toestellen.
Een theoretische oefening op de uitvoeringstijd bij nog grotere aantallen toestellen leerde ons dat de impact van extra te bevragen toestellen beperkt genoeg blijft
om de retriever te kunnen inzetten op zeer grote netwerken.

Ook het geheugen- en bandbreedtegebruik blijft zeer minimaal bij het bevragen van een groot aantal toestellen.
Bij het CPU-gebruik konden we wel nog opmerken dat er bij het beheren van het aantal threads nog verbetering mogelijk is.


\todo[inline,caption={}]{

\begin{itemize}
	\item Doel masterproef herhalen
	\item Resultaten bespreken.
	\item Door gebruik van bulk en log4net \ldots
	\item Geslaagd in opzet?
	\item 
	\item Samenvatting van de conclusies van de tests.
	\item Verdere optimalisaties.
	\item 
	\item Vraag: is de retriever uiteindelijk schaalbaar genoeg?
	\item 
	\item antwoord op centrale vraag, korte samenvatting, geen nieuwe elementen \ldots
\end{itemize}

}