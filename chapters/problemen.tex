\chapter{Problemen}

\todo[inline]{Zou dit niet beter een apart hoofdstuk zijn?}

\todo[inline,caption={}]{

\begin{itemize}

	\item Kolombreedte OID's (en andere? Hostnaam?) te klein
	\item endOfMibView exceptie niet ondersteund door SNMP Data Retriever \\
		Zie pg. 7, verslag week 9-10
	\item Grootschalige testopstelling op de Virtual Wall \\
		Probleem met de virtualisatie (beta achtige feature, werkt in de praktijk niet zo goed)
		Problemen met netwerkopstelling/adapters voor virtuele nodes
	
\end{itemize}

Niet opgelost/niet belangrijk:

\begin{itemize}

	\item DDOS-bescherming bij iMinds
	\item Permissieproblemen met snmpd.	$ \rightarrow $ IP range beperking~
	\item Omdat mijn reactietijden.pl script maar een commando uitvoert, hebben we voor het benchmarken van bulkrequests die manueel opgesteld worden een ander script nodig.
	Ik heb me hiervoor gebaseerd op m'n benchmarking script voor Windows die zelf de tijden meet (omdat ik de /usr/bin/time van Linux niet heb in Windows).
	Deze maakt gebruik van een CPAN module. Om zeker te zijn dat er geen grote verschillen tussen de timing-methoden zitten, heb ik nog een derde manier geprobeerd.
	Ik maak gebruik van HiRes time en hou start- en eindtijd bij en geef het verschil terug. De drie timingmethoden komen dicht genoeg bij elkaar in de buurt om te concluderen
	dat er geen timingverschillen optreden bij het gebruik van verschillende timingmethoden en ik dus de tijden kan vergelijken. \\
	Zie ook meerderekolommen - Vergelijk timings.pl en Verschillen in benchmark timings.txt.
	\item Totaal aantal objecten dat een agent aanbiedt.
	\item Gaten in tabellen, zie random notes.txt
	\item Problemen met verouderde softwarepakketten op Debian
	
\end{itemize}

}


\section{Ondersteuning voor lange OID's}
\label{probleem-lange-oids}

Een probleem waar we al redelijk vroeg op stootten, was het voorkomen van lange \glspl{oid} met SQL-fouten tot gevolg.
De grote boosdoener hier was de \textit{inetCidrRouteTable} (1.3.6.1.2.1.4.24.7) die zeer lange indexen heeft,
zoals te zien is in \cref{lst-inetCidrRouteEntry}.

\begin{lstlisting}[language=asn.1, float=h, caption={Definitie van een inetCidrRouteEntry}, label=lst-inetCidrRouteEntry]
inetCidrRouteEntry OBJECT-TYPE 
	 SYNTAX InetCidrRouteEntry
	 MAX-ACCESS not-accessible
	 STATUS current
	 DESCRIPTION "A particular route to a particular destination, under a 
            particular policy (as reflected in the 
            inetCidrRoutePolicy object). 
 
            Dynamically created rows will survive an agent reboot. 
 
            Implementers need to be aware that if the total number 
            of elements (octets or sub-identifiers) in 
            inetCidrRouteDest, inetCidrRoutePolicy, and 
            inetCidrRouteNextHop exceeds 111, then OIDs of column 
            instances in this table will have more than 128 sub- 
            identifiers and cannot be accessed using SNMPv1, 
            SNMPv2c, or SNMPv3."
	 INDEX { inetCidrRouteDestType, inetCidrRouteDest, inetCidrRoutePfxLen, inetCidrRoutePolicy, inetCidrRouteNextHopType, inetCidrRouteNextHop } 
 	::= { inetCidrRouteTable 1  }
\end{lstlisting}

Die index bestaat onder andere uit een IP-adres dat een IPv4- of een IPv6-adres kan zijn.
Zoals je weet zijn IPv6-adressen vrij lang, 128 bits om precies te zijn.
Een voorbeeld van een \gls{oid} uit de inetCidrRouteTable zie je in \cref{lst-lange-oids}.

\begin{lstlisting}[float=h, caption={Tekstuele en numerieke notatie van een \gls{oid} uit inetCidrRouteTable}, label=lst-lange-oids]
IP-FORWARD-MIB::inetCidrRouteIfIndex.ipv6."fe:80:00:00:00:00:00:00:0a:00:27:ff:fe:6d:bd:c5"
	.128.1.7.ipv6."00:00:00:00:00:00:00:00:00:00:00:00:00:00:00:00" = INTEGER: 1

1.3.6.1.2.1.4.24.7.1.7.2.16.254.128.0.0.0.0.0.0.10.0.39.255.254.109.189.197
	.128.1.7.2.16.0.0.0.0.0.0.0.0.0.0.0.0.0.0.0.0 = INTEGER: 1
\end{lstlisting}

Omdat de resultaattabel die de originele versie van de \nwmretriever{} aanmaakte slechts 55 karakters voorzag voor een \gls{oid} resulteerde dit natuurlijk
in SQL-fouten omdat de de \glspl{oid} veel langer waren zoals ook te zien is in \cref{lst-sql-error}.


\begin{lstlisting}[float=h, caption={SQL-fout bij te lange \glspl{oid}}, label=lst-sql-error]
2014/02/24 14:28:54.92	Error		[21/311MB]	Failed to insert result row: 
ComfunSQLConnectionException: Error executing SQL statement: 'replace into results values(@devname,'1.3.6.1.2.1.4.34.1.5.2.16.254.128.0.0.0.0.0.0.10.0.39.255.254.109.189.197', @attrname,@value,'2014-02-24 14:28:54','OK','2014-02-24 14:28:54')' with parameters '@attrname RFC1213-MIB.ip System.String @devname debian-vm-01 System.String @value 1.3.6.1.2.1.4.32.1.5.2.2.16.254.128.0.0.0.0.0.0.0.0.0.0.0.0.0.0.64 System.String' using connectionstring 'database=snmpdb;data source=localhost;user id=xxxxx;password=xxxxx;port=3306;old syntax=yes'. ---> MySql.Data.MySqlClient.MySqlException: #22001Data too long for column 'OID' at row 1
\end{lstlisting}

De oplossing is gelukkig eenvoudig: met het vergroten van de kolombreedte voor \glspl{oid} in de resultaattabel is het probleem opgelost.
Om dat te doen passen we simpelweg de SQL-query aan in de \nwmretriever{} die de tabel aanmaakt.
We nemen een grote marge voor de kolombreedte en vergroten ze naar 4096 karakters om toekomstige problemen te vermijden en omdat brede kolommen geen grote impact hebben.

Reeds bestaande tabellen moeten echter wel ofwel manueel aangepast worden om de kolom te verbreden,
ofwel kan men de tabel verwijderen en de \nwmretriever{} een nieuwe laten aanmaken als de reeds opgeslagen resultaten ook verwijderd mogen worden.


\section{Ondersteuning voor de endOfMibView-exceptie}
\label{probleem-endofmibview-exceptie}

Zoals we gezien hebben in de berichtstructuur van een SNMP-bericht in \cref{snmp-berichtstructuur},
zijn er twee velden voorzien in een SNMP-bericht om aan te geven dat er zich een fout heeft voorgedaan: \textit{Error} en \textit{Error Index}.
Het eerste veld geeft met een statuscode aan of er zich een fout heeft voorgedaan en zoja, dewelke.
Het tweede veld bevat in geval van een fout een wijzer naar het object dat de fout veroorzaakt heeft

Een van de nieuwe dingen die SNMPv2c introduceerde zijn SNMP-excepties.
Er bestaan drie zulke excepties: \textit{EndOfMibView}, \textit{NoSuchInstance} en \textit{NoSuchObject}.
Het bijzondere aan deze SNMP-excepties is dat ze toegevoegd worden aan de objectlijst (Varbind List) van een bericht
en het errorveld niet wijzigen als ze zich voordoen\cite{endOfMibView-error-status}.

De reden daarvoor is zodat een exceptie zich kan voordoen samen met andere objecten die wel correct opgehaald zijn.
Herinner je dat je met GET- en GETNEXT-requests meerdere gegevens kunt ophalen,
maar dat als er zich een fout voordoet je helemaal geen gegevens terug zal krijgen (\cref{meerdere-gegevens-ophalen-met-GET-en-GETNEXT}).
Met SNMPv2c werd dit probleem dus opgelost.

De endOfMibView-exceptie doet zich voor bij een SNMP walk\footnote{
	Zowel bij gebruik van GETNEXT- en GETBULK-requests indien gebruik gemaakt wordt van SNMPv2c of hoger.
	Bij SNMPv1 met GETNEXT-requests wordt de foutcode \textit{noSuchName} gebruikt\cite{snmp-exceptions-v2c-to-v1}.
} wanneer er geen verdere data meer op te halen is\cite{snmp-exceptions-explained}.
Wij kwamen deze exceptie tegen bij het overlopen van een ganse SNMP-agent, waarbij er op het einde natuurlijk geen gegevens meer zijn en de exceptie opgegooid wordt.

Omdat de originele versie van de \nwmretriever{} enkel gebruik maakte van GET- en GETNEXT-requests was SNMPv1 voldoende.
Bovendien is het overlopen van een ganse SNMP-agent vrij uitzonderlijk dus kwam men nooit in aanraking met deze (en andere) excepties.
Het resultaat is dat er voor het detecteren van fouten enkel gekeken werd naar het erroveld en er dus geen rekening gehouden werd met excepties.

In de originele versie had dit geen gevolgen omdat excepties zich toch niet voordoen in SNMPv1.
Bij de nieuwe versie echter zal hier wel op moeten gecontroleerd worden vermits er gebruik gemaakt wordt van SNMPv2c.

De manier om dit aan te pakken is vrij eenvoudig.
Men controleert net zoals voorheen nog steeds het errorveld, maar bij het overlopen van de objecten in het antwoordbericht
zal men ook moeten controleren of er een exceptie aanwezig is en de nodige acties ondernemen.
Bij een endOfMibView-exceptie zal men moeten de SNMP walk staken want ze is immers ten einde gelopen.


\section{''DOS-bescherming'' op intern netwerk iMinds}
\label{probleem-dos-bescherming}

\todo[inline]{Betere titel? Alhoewel de oorzaak nooit is vastgelegd...}

\section{Grootschalige opstelling op de Virtual Wall m.b.v. virtualisatie}
\label{probleem-virtualisatie-vwall}

\todo[inline]{Betere/kortere titel?}