\chapter*{Abstract}
\addcontentsline{toc}{chapter}{Abstract}

\section*{Nederlands}

SNMP biedt een uniforme manier aan om informatie op te halen over netwerkcomponenten, ongeacht de fabrikant.
Er zijn veel toepassingen maar enkele belangrijke zijn inventarisatie en configuratie- en performantiebeheer van netwerkcomponenten zoals routers en switches.
In een grootschalig netwerk is het ondenkbaar dat er geen gebruik gemaakt wordt van tools die van deze gegevens gebruik maken.

In dit werk wordt onderzocht hoe een bestaande tool die momenteel gebruikt wordt om kleine tot middelgrote netwerken te bevragen via SNMP,
kan ingezet worden op grootschalige netwerken.
Daarvoor wordt een diepgaande analyse uitgevoerd van alle aspecten die hierbij betrokken zijn,
van software- tot netwerkniveau om problemen die de schaalbaarheid beïnvloeden bloot te leggen.

Eenmaal die problemen vastgesteld zijn worden mogelijke oplossingen besproken, geïmplementeerd en wordt de effectiviteit ervan geëvalueerd.


Kernwoorden: SNMP, bevraging, retrieval, retriever, schaalbaarheid, Net-SNMP, analyse, performantie, berichtstructuur, profiler, netwerkvertraging, bulk, optimalisatie


\section*{English}

SNMP offers a uniform way to to retrieve information about network components, regardless of the manufacturer.
There are many applications, but a few important ones are taking stock, configuration and performance management of network components such as routers and switches.
In a large scale network it is unthinkable that tools which make use of this information would not be used.

This work researches how an existing tool that's currently used to query small to medium sized networks via SNMP,
can be used in large scale networks.
To that end, a thorough analysis is done of all aspects involved, from software to network level, to find problems that affect the scalability.

Once those problems are identified, solutions are discussed, implemented and their effectiveness evaluated.


Keywords: SNMP, querying, retrieval, retriever, scalability, Net-SNMP, analysis, performance, message structure, profiler, network latency, bulk, optimisation